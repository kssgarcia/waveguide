% $Header: /cvsroot/latex-beamer/latex-beamer/solutions/generic-talks/generic-ornate-15min-45min.en.tex,v 1.5 2007/01/28 20:48:23 tantau Exp $

\documentclass{beamer}

% This file is a solution template for:

% - Giving a talk on some subject.
% - The talk is between 15min and 45min long.
% - Style is ornate.



% Copyright 2004 by Till Tantau <tantau@users.sourceforge.net>.
%
% In principle, this file can be redistributed and/or modified under
% the terms of the GNU Public License, version 2.
%
% However, this file is supposed to be a template to be modified
% for your own needs. For this reason, if you use this file as a
% template and not specifically distribute it as part of a another
% package/program, I grant the extra permission to freely copy and
% modify this file as you see fit and even to delete this copyright
% notice.


\mode<presentation>
{
  \usetheme{Madrid}
  % or ...
  \setbeamercovered{transparent}
  \setbeamertemplate{theorems}[numbered]
  \setbeamertemplate{figures}[numbered]
  % or whatever (possibly just delete it)
}


%\usepackage[spanish]{babel}
% or whatever
\usepackage{graphicx}
\usepackage[latin1]{inputenc}
% or whatever

\usepackage{times}
\usepackage[T1]{fontenc}
% Or whatever. Note that the encoding and the font should match. If T1
% does not look nice, try deleting the line with the fontenc.
\usepackage{amsmath,amsthm}
\usepackage{amssymb}
\usepackage{amsmath}

\usepackage{graphicx}
\usepackage{float}
\usepackage{amsmath}
\usepackage{amsthm}
\usepackage{amssymb}
\usepackage{amscd}
\usepackage{amsfonts}
\usepackage{makeidx}
\usepackage{enumerate}      
\usepackage{IEEEtrantools}
%\usepackage{cases}
%\usepackage{cases}
\usepackage{mathrsfs}

%\usepackage{amsmath,empheq}

\usepackage{color}

%%%%%%%%%%%%%%%%%%%%%%%%%%%%%%%%%%%%%%%%%%%%%%%
%\usepackage{showkeys}
\usepackage{amsmath}

\usepackage{graphicx}
\usepackage{float}
\usepackage{amsmath}
\usepackage{amsthm}
\usepackage{amssymb}
\usepackage{amscd}
\usepackage{amsfonts}
\usepackage{makeidx}
\usepackage{enumerate}      
\usepackage{IEEEtrantools}
%\usepackage{cases}
%\usepackage{cases}
\usepackage{mathrsfs}

%\usepackage{amsmath,empheq}


\usepackage[latin1]{inputenc}


%%%%%%%%%%%%%%%%%%%%%%%%%%%%%%%%%%%%%%%%%%%%%
\usepackage [all]{xy}
%%%%%%%%%%%%%%%%%%%%%%%%%%%%%%%%%%%%%%%%%%
\usepackage{cases}

\usepackage{amsmath}

\usepackage{cancel}

\usepackage{mathrsfs}



\newtheorem{teo}{Theorem}[section]
    \newtheorem{lem}[teo]{Lemma}
    \newtheorem{prop}[teo]{Proposition}
    \newtheorem{coro}[teo]{Corollary}
    \newtheorem{defn}[teo]{Definition}
    \newtheorem{obs}[teo]{Remark}
    \newtheorem{exa}[teo]{Example}
%\newtheorem{theorem}[section]{Teorema}
%\renewcommand{\thetheorem}{\arabic{section}.\arabic{theorem}}
%\renewcommand{\theequation}{\thesection.\arabic{equation}}  %%% PARA ENUMERAR LAS %ECUACIONES POR SECCIÓN CON NÚMEROS ARÁBIGOS
\newcommand{\re}{\ref}
\newcommand{\fr}{\frac}
\newcommand{\ci}{\cite}
\newcommand{\ov}{\overline}
    \newtheorem*{dem}{\textsc{Proof}}
    \newcommand{\bdem}{\begin{dem}}
    \newcommand{\edem}{\end{dem}}
%%%%%%%%%%%%%%
     \newcommand{\be}{\begin{equation}}
    \newcommand{\ee}{\end{equation}}
     \newcommand{\ba}{\begin{array}}
    \newcommand{\ea}{\end{array}}
\newcommand{\beqn}{\begin{eqnarray}}
    \newcommand{\eeqn}{\end{eqnarray}}
    \newcommand{\bl}{\begin{lem}}
    \newcommand{\el}{\end{lem}}
    \newcommand{\bp}{\begin{prop}}
    \newcommand{\ep}{\end{prop}}
\newcommand{\ds}{\displaystyle}
     \newcommand{\la}{\label}
     \newcommand{\al}{\alpha}
      \newcommand{\Ga}{\Gamma}
      \newcommand{\cC}{\cal C}
    \newcommand{\De}{\Delta}
     \newcommand{\om}{\omega}
    \newcommand{\vp}{\varphi} \newcommand{\ga}{\gamma}
    \newcommand{\R}{\mathbb{R}}
    \newcommand{\C}{\mathbb{C}}
     \newcommand{\Z}{\mathbb{Z}}
     \newcommand{\no}{\noindent}
     \newcommand{\Res}{{\rm Res}}
     \newcommand{\vare}{\varepsilon}
   
   
   %%%%%%%%%%%%%%%%%%%%%%%%%%%%%%%%%%%%%%%%%%%%%%%%%%%%%%%%%%%%%% 
   \providecommand{\abs}[1]{\lvert#1\rvert}
\providecommand{\norm}[1]{\lVert#1\rVert}
 
    
    
   %%%%%%%%%%%%%%%%%%%%%%%%%%%%%%%%%%%%%%%%%%%%%%%%%%%%%%%%%%% 
    
    
    \IEEEyessubnumber

\title[Trapped Modes]{Discrete and embedded trapped modes in a quantum waveguide with a small obstacle}
\author[Zhevandrov et al.]{
  P. Zhevandrov, A.  Merzon, M.I. Romero Rodr\'iguez
  }
\institute[UMSNH, UMNG]{
{Facultad de Ciencias F\'{\i}sico-Matem\'aticas, UMSNH, Morelia, Mexico}\\
{Instituto de  F\'{\i}sica y Matem\'aticas, UMSNH, Morelia, Mexico}\\
{Universidad Militar ``Nueva Granada'', Bogot\'a, Colombia}
}
\date[2025]{XIII Taller de Geometr\'\i a y Sistemas Din\'amicos}


\subject{Talks}
% This is only inserted into the PDF information catalog. Can be left
% out.



% If you have a file called "university-logo-filename.xxx", where xxx
% is a graphic format that can be processed by latex or pdflatex,
% resp., then you can add a logo as follows:

%\pgfdeclareimage[height=1 cm]{university-logo}{university-logo-filename}
%\logo{\pgfuseimage{university-logo}}



% Delete this, if you do not want the table of contents to pop up at
% the beginning of each subsection:
%\AtBeginSubsection[]
%{
% \begin{frame}<beamer>{Outline}
%    \tableofcontents[currentsection,currentsubsection]
%  \end{frame}
%}


% If you wish to uncover everything in a step-wise fashion, uncomment
% the following command:

%\beamerdefaultoverlayspecification{<+->}


\begin{document}
\begin{frame}
  \titlepage
\end{frame}

\begin{frame}{Outline}
  \tableofcontents
  % You might wish to add the option [pausesections]
\end{frame}




% Since this a solution template for a generic talk, very little can
% be said about how it should be structured. However, the talk length
% of between 15min and 45min and the theme suggest that you stick to
% the following rules:

% - Exactly two or three sections (other than the summary).
% - At *most* three subsections per section.
% - Talk about 30s to 2min per frame. So there should be between about
%   15 and 30 frames, all told.
\section{Formulation}
\begin{frame}{Formulation}

\begin{figure}[htbp]
\centering
\includegraphics[scale=0.18]{Gam_d.pdf}
%\caption{Small obstacle in a waveguide}\label{Gam_d}
\end{figure}
\centerline{Small obstacle in a waveguide}
\end{frame}
\begin{frame}
Consider the following boundary value problem:
\begin{align}\label{PD1}
-\Delta u&=k^2 u,\quad x,y\in\Omega,
\\\nonumber\\\label{PD2}
\left.u\right\vert_{\Gamma_{\pm}}&=0,\quad \left.\frac{\partial u}{\partial n}\right\vert_{\gamma}=0,
\end{align}
where $k^2$ is a spectral parameter, 
\end{frame}
\begin{frame}
$\gamma=\Big\lbrace x=\varepsilon X(t),~y=a+\varepsilon Y(t)\Big\rbrace$ is the boundary of the obstacle and $\Omega$ is the interior of the strip $\lbrace-b<y<b\rbrace$ without the obstacle (i.e., the exterior of $\gamma$), $\Gamma_{\pm}=\Big\lbrace x\in\R,~y=\pm b\Big\rbrace$.
We assume that $|a|<b$, $X(t)$ and $Y(t)$ are $2\pi$-periodic,  $C^{\infty}$-functions with zero mean, $\int\limits_{-\pi}^{\pi} X(t)\;dt=\ds\int\limits_{-\pi}^{\pi} Y(t)\;dt=0$. 

Trapped modes are, by definition, {\it nontrivial solutions of (\ref{PD1}), (\ref{PD2}) which decay at infinity} (i.e. as $|x|\to\infty$; more precisely, they belong to the Sobolev space $H_1(\Omega)$). The corresponding values of $k^2$ are eigenvalues of (\ref{PD1}), (\ref{PD2}). 
\end{frame}
\begin{frame}{Continuous spectrum}
\begin{figure}[htbp]
\centering
\includegraphics[scale=0.70]{spectrumWG.png}
%\caption{Small obstacle in a waveguide}\label{Gam_d}
\end{figure}
\centerline{Continuous spectrum, $\Lambda_n=\pi^2n^2/4b^2$}
\end{frame}
\section{Main results}
\begin{frame}{Main results}
Consider the exterior Neumann problem on the plane
\begin{equation}\label{Dlta}
\Delta\Psi=0~~{\rm in}~~\Omega_0,\quad \ds\frac{\partial \Psi}{\partial n}\Bigg\vert_{\Gamma}=n_2,\quad \nabla\Psi\to0~~{\rm as}~~r=\sqrt{x^2+y^2}\to\infty,
\end{equation} 
where the contour $\Gamma$ is the ``inflated'' contour $\gamma$, $\Gamma=\big\lbrace x=X(t), y=Y(t)\big\rbrace$, $\Omega_0$ is the exterior of this contour on the plane, $\ds\frac{\partial}{\partial n}$ is the derivative along the inward-looking normal to $\Gamma$, $n_2=\ds\frac{\dot{X}}{\sqrt{\dot{X}^2+\dot{Y}^2}}$ is its second component. Problem (\ref{Dlta}) describes the vertical flow of an unbounded fluid past the contour $\Gamma$. The solution of this problem is unique up to an additive constant  and
\begin{equation}\label{Psi}
\Psi={\rm const}-\mu\ds\frac{y}{r^2}-\nu\ds\frac{x}{r^2}+O\left(\ds\frac{1}{r^2}\right)~~{\rm as}~~r\to\infty.
\end{equation} 

\end{frame}
\begin{frame}
We can assume that const in (\ref{Psi}) is equal to $0$. The constants $\mu$ and $\nu$ are called {\it strengths of the vertically and horizontally oriented  dipoles} corresponding to (\ref{Dlta}), and %\cite{Neumann},
\begin{equation}\label{mu}
\mu=\ds\frac{1}{2\pi}\Bigg(S+\int\limits_{\Gamma} n_2\Psi\;dl\Bigg),\quad\nu=\ds\frac{1}{2\pi}\int\limits_{\Gamma} n_1\Psi\;dl,
\end{equation}
where $S$ is the area bounded by $\Gamma$, $dl$ is the element of the arclength and $n_{1,2}$ are the components of the inward-looking normal to $\Gamma$. The constant $\mu$ is always positive.\\
Denote
\begin{equation}\label{a_0^ast}
a_0^{\ast}=\frac{2b}{\pi}\arctan\sqrt{\ds\frac{S}{2\pi \mu}}.
\end{equation}
\end{frame}
\begin{frame}{Interval $[0,\Lambda_1)$}
 \begin{teo}\label{sigma_alpha}
For sufficiently small $\varepsilon$, there exists a value of $a=a^{\ast}=a_0^{\ast}+O(\varepsilon)$ such that (\ref{PD1}), (\ref{PD2}) possesses a unique trapped mode on $[0,\Lambda_1)$ if $a>a^{\ast}$. If $a\leq a^{\ast}$, there are no trapped modes on this interval. The eigenvalue, when it exists, is analytic in $\varepsilon$ and $\varepsilon_1=\varepsilon\ln\varepsilon$, and is given by $k^2=\Lambda_1-\sigma^2$, where   
\begin{equation}\label{sigmaO(var)}
\sigma=\varepsilon^2\frac{\pi^2}{4b^3}\Bigg(\pi\mu\sin^2\frac{\alpha}{2}-\frac{1}{2}S \cos^2\frac{\alpha}{2}\Bigg)+O(\varepsilon^3\ln\varepsilon),\quad \alpha=\frac{\pi a}{b}.
\end{equation}
\end{teo}
\begin{obs}
Formula (\ref{sigmaO(var)}) is known, our contribution consists in the proof of the analyticity of $\sigma$ in $\varepsilon, \varepsilon_1$ and the uniqueness of it. 
\end{obs}
%\end{teo}
\end{frame}
\begin{frame}{Example}
\begin{exa}
Formula (\ref{sigmaO(var)}) shows that if the expression 
\begin{equation*}
-S\cos^2\alpha +2\pi\mu\sin^2\alpha
\end{equation*}
is positive, then there exists a discrete eigenvalue of our problem given by $k^2=\ds\frac{\pi^2}{4b^2}-\sigma^2$. For example, for a circle of radius $r_0$ (i.e., $S=\pi r_0^2,~\mu=r_0^2$) this will be the case if
$$
\ds\frac{1}{2}-\ds\frac{3}{2}\cos2\alpha>0
$$
which is true if $a$, for example, is close to $b$ (the obstacle is close to the upper boundary of the guide). 
\end{exa}
\end{frame}
\begin{frame}{Interval $\Lambda_1\leq k^2<\Lambda_2$}
\begin{teo}\label{a^{ast}+}
For sufficiently small $\varepsilon$, the following statements are valid:
\begin{enumerate}
%\item[(i)] There are no eigenvalues on the interval $[\Lambda_1,\Lambda_2-\sigma_0]$ with any fixed $\sigma_0>0$. 
\item[(i)] If $\nu\neq0$ or $|a|\geq\delta>0$ with $\delta$ independent of $\varepsilon$, then there are no eigenvalues on $[\Lambda_1,\Lambda_2)$.
\item[(ii)] If $\Gamma$ is symmetric with respect to the $x$-axis, then there exists a unique eigenvalue on $[\Lambda_1,\Lambda_2)$ given by $k^2=\Lambda_2-\sigma^2$, where
\begin{equation}\label{sigma_O(varepsilon)}
\sigma=\varepsilon^2\;\ds{(\pi/b)^3} \mu+O(\varepsilon^3\ln\varepsilon).
\end{equation}
\item[(iii)] If $\Gamma$ is symmetric with respect to the $y$-axis, then there exists a unique eigenvalue on $[\Lambda_1,\Lambda_2)$ if $a=\varepsilon a_1+O(\varepsilon^2)$ given by  (\ref{sigma_O(varepsilon)}), where $a_1$ is given by 
\begin{equation}\label{fa1}
a_1=\textstyle(2(2S+\pi\mu))^{-1}\textstyle\int\limits_{-\pi}^{\pi} (Y\dot{X}-3X\dot{Y}) \left(Y-\Psi(X,Y)\right)\;dt.
\end{equation}  
\end{enumerate}
\end{teo}
\end{frame}

\begin{frame}


%\newpage
\begin{figure}[htbp]
\centering
\includegraphics[scale=0.15]{FIGpa_33.pdf}
\end{figure}
\centerline{Contour $\Gamma$ for an obstacle symmetric with respect to $x$-axis} 
\centerline{($a=0$, $Y(t)$ is odd and $X(t)$ is even; in this case $\nu=0$ automatically)}
\end{frame}
\begin{frame}
%\newpage
\begin{figure}[htbp]
\centering
\includegraphics[scale=0.15]{FigPARA_1.pdf}
\end{figure}
\centerline{Contour $\Gamma$ for an obstacle symmetric with respect to $y$-axis} 
\centerline{($Y(t)$ is even and $X(t)$ is odd; in this case we also have $\nu=0$)}
\end{frame}

\begin{frame}{Example}
\begin{exa}\label{X}
In the case of a slightly perturbed circle, it is possible to calculate explicitly all the objects entering formula (\ref{fa1}). For example, if $X(t)=\sin t-\ds\frac{\beta}{2} \sin 2t,~ Y(t)= -\cos t+\ds\frac{\beta}{2}\cos2t$, $\beta\ll 1$,  $a_1=-\beta/12+O(\beta^2)$.
\end{exa}

\end{frame}
\begin{frame}
\begin{obs}
The existence of the eigenvalue in $(ii)$ (even when $\varepsilon$ is not small) can be proven by means of restricting problem (\ref{PD1}), (\ref{PD2}) to solutions odd in  $y$  (the restriction of problem (\ref{PD1}), (\ref{PD2}) to odd in $y$ functions removes the interval $[\Lambda_1,\Lambda_2)$ from the continuous spectrum) thus reducing the problem of embedded eigenvalues to the discrete spectrum of the restricted problem.\\Our contribution consists mainly in $(iii)$ (the restriction to odd or even in $x$ functions does not remove the interval $[\Lambda_1,\Lambda_2)$ from the continuous spectrum, thus the eigenvalue in this case is truly embedded) and the proof of uniqueness and analyticity of $\sigma$. 
\end{obs}

\begin{obs}
The uniqueness of the eigenvalues is due exclusively to the smallness of the obstacle. Similarly to the one-dimensional Schr\"odinger equation with a shallow potential well, the number of eigenvalues can augment when the obstacle becomes large (just as in the case of deep potential well).
\end{obs}
\end{frame}

\section{Schr\"odinger equation}



%\begin{frame}{Make Titles Informative.}
%
%  You can create overlays\dots
%  \begin{itemize}
%  \item using the \texttt{pause} command:
%    \begin{itemize}
%    \item
%      First item.
%      \pause
%    \item
%      Second item.
%    \end{itemize}
%  \item
%    using overlay specifications:
%    \begin{itemize}
%    \item<3->
%      First item.
%    \item<4->
%      Second item.
%    \end{itemize}
%  \item
%    using the general \texttt{uncover} command:
%    \begin{itemize}
%      \uncover<5->{\item
%        First item.}
%      \uncover<6->{\item
%        Second item.}
%    \end{itemize}
%  \end{itemize}
%\end{frame}


\begin{frame}{Schr\"odinger Equation}
$$\hat K_{Sch}\psi=-\psi''+\varepsilon V(x)\psi=E\psi,\qquad \varepsilon\ll1,$$
$$E=-\sigma^2,\qquad \psi\simeq\exp(-\sigma|x|),\qquad \sigma\to0,\varepsilon\to0$$
\centerline{\includegraphics [width=8cm ,height=5cm ]{swell.png}}
\end{frame}

\begin{frame}{Spectrum}
\centerline{\includegraphics[width=8cm, height=5cm ]{spec.png}}
\centerline{The distance between the eigenvalue} \centerline{ and the continuous spectrum is $\sigma^2$.}
$$\sigma\sim\varepsilon$$
\end{frame}




\begin{frame}{Fourier transform}
\centerline{$E=-\sigma^2,\qquad\sigma>0$.}
\begin{equation}\label{SE}
-\psi''+\varepsilon V\psi=-\sigma^2\psi,
\end{equation}
Apply Fourier transform to (\ref{SE}):\begin{equation}\label{TFES1}
(p^2+\sigma^2)\tilde{\psi}(p)=-\frac{\varepsilon}{2\pi}\int \tilde{V}(p-p')\tilde{\psi}(p')dp',
\end{equation}
\end{frame}

\begin{frame}{Form of the solution}
\centerline{$\psi\sim e^{-\sigma|x|},\qquad \tilde\psi(p)\sim\delta(p)$,}
\centerline{$\tilde\psi(p)=\int e^{-ipx}\psi(x)\,dx=\mathcal{F}_{x\to p}\psi(x)$}
$$(p^2+\sigma^2)\tilde\psi=-\frac\varepsilon{2\pi}\int\tilde V(p-p')\tilde\psi(p')\,dp'\simeq C\tilde V(p)$$
$$\tilde\psi\sim C \frac{\tilde V(p)}{p^2+\sigma^2}=\frac{A(p,\varepsilon)}{p^2+\sigma^2}$$
$$A(p,\varepsilon)=A_0(p)+\varepsilon A_1(p)+\dots,\qquad \sigma=\varepsilon\sigma_1+\varepsilon^2\sigma_2+\dots$$
\end{frame}

\begin{frame}{Exact solution}
Look for the solution of (\ref{TFES1}) in the form:
\begin{equation}\label{Eqtphi}
\tilde{\psi}(p,\varepsilon)=\frac{A(p,\varepsilon)}{p^2+\sigma^2},
\end{equation}
Substituting (\ref{Eqtphi}) in (\ref{TFES1}), we obtain:
\begin{equation}\label{EqAS}
A(p,\varepsilon)=-\frac{\varepsilon}{2\pi}\int \frac{\tilde{V}(p-p')A(p',\varepsilon)}{p'^2+\sigma^2}dp'.
\end{equation}
\emph{Note that for $\sigma=0$ (\ref{EqAS}) has a singularity at $p'=0$.}
\end{frame}


\begin{frame}{Exact solution}
Introduce
\begin{align}\label{ECG}\gamma:=(-\infty,-1] \cup \{p:\,|p|=1, \Im p>0\} \cup[1,\infty).\end{align}

\centerline{\includegraphics[width=8cm ,height=2.5cm ]{gamma.png}}

\end{frame}

\begin{frame}{Exact solution}
Apply the Cauchy Residue Theorem  to the right-hand side of (\ref{EqAS}):
\begin{equation}\label{EqAS1}
A(z)=-\frac{\varepsilon}{2\pi}\int_\gamma \frac{\tilde{V}(z-\zeta)A(\zeta)}{\zeta^2+\sigma^2}d\zeta-\frac{\varepsilon}{2\sigma} \tilde{V}(z-i\sigma)A(i\sigma),
\end{equation}
$\tilde{V}(\zeta)$ is the analytic continuation of $\tilde{V}(p)$ to the complex plane.
\end{frame}

\begin{frame}{Exact solution}
 Define the integral operator $T_\sigma$  acting on the Banach space of functions analytic in a strip containing the real axis with the standard sup-norm by
\begin{align*}
[T_\sigma \varphi(\zeta)](z)=\frac{1}{2\pi}\int_\gamma \frac{\tilde{V}(z-\zeta)\varphi(\zeta)}{\zeta^2+\sigma^2}d\zeta,
\end{align*}

 and write (\ref{EqAS1}) in terms of $T_\sigma$:
\begin{equation*} [(1+\varepsilon T_\sigma)A(\zeta)](z)=-\frac{\varepsilon}{2\sigma}\tilde{V}(z-i\sigma)A(i\sigma).\end{equation*}



\end{frame}

\begin{frame}{Explicit solution}
$T_\sigma$ is bounded,
$\varepsilon T_\sigma$ is small, for this reason (\ref{EqAS1}) gives
\begin{equation}\label{EqAS2}
A(z)=-\frac{\varepsilon}{2\sigma}A(i\sigma)[(1+\varepsilon T_{\sigma})^{-1}\tilde{V}(\zeta-i\sigma)](z),
\end{equation}
$(1+\varepsilon T_{\sigma})^{-1}=\sum_{n=0}^{\infty}(-1)^n\varepsilon^n T_\sigma^n,$ $T_\sigma^0\equiv 1$ (the Neumann series).
\end{frame}


\begin{frame}{Explicit solution}
 Evaluate (\ref{EqAS2}) at $z=i\sigma$,
multiply by $\sigma$  equation (\ref{EqAS2}) and divide by $A(i\sigma)$. We obtain the \emph{\textbf{secular equation}} for $\sigma:$
  \begin{equation}\label{eqsES}
    \sigma=-\frac{\varepsilon}{2} [(1+\varepsilon T_{\sigma})^{-1}\tilde{V}(\zeta-i\sigma)](i\sigma)\sim-\frac\varepsilon2\int V(x)\,dx.
  \end{equation}
\end{frame}

\section{Reduction to Integral Equations}



\begin{frame}{Waveguide - Reduction to Integral Equations}
We derive the system  for the boundary values of $u$. We have, by the Green formula applied to the domain $\Omega$, 
\begin{equation}\label{GF}
u(\xi,\eta)=-\ds\int\limits_{\Gamma_{+}+\Gamma_{-}+\gamma} G\Big(x-\xi,y-\eta\Big)\ds\frac{\partial u}{\partial n} dl+\ds\int\limits_{\Gamma_{+}+\Gamma_{-}+\gamma} u\;\ds\frac{\partial G(x-\xi,y-\eta)}{\partial n} dl,\quad(\xi,\eta)\in\Omega,\quad(x,y)\in\partial\Omega
\end{equation}
$dl$ is the element of the arclength of $\partial\Omega$ at the point $(x,y)$, $G(x,y)$ is any fundamental solution of (\ref{PD1}) bounded at infinity, $\Delta G+k^2 G=\delta(x,y)$ and $\partial/\partial n$ is the derivative along the exterior normal to $\Omega$.
 We use the fundamental solution of the form $G(x,y)=\ds\frac{1}{4i}\;H_{0}^{(1)}(k r)$ in the equation on $\gamma$, and in equations on $\Gamma_\pm$  we use the fundamental solution of the form  $G(x,y)=\ds\frac{1}{4}\;N_{0}(k r)$; here $r=\sqrt{x^2+y^2}$ and $H_0^{(1)}$ is the Hankel function of the first kind and $N_0$ is the Neumann function.
\end{frame}

\begin{frame}
We  reduce our problem to three integral equations for the functions $\varphi(x)$, $\psi(x)$ and $\theta(t)$, where
\begin{align*}
 \varphi(x)=\ds\frac{\partial u}{\partial n}\Big\vert_{\Gamma_{+}}=u_{y}\Big\vert_{\Gamma_{+}},\qquad \psi(x)=\ds\frac{\partial u}{\partial n}\Big\vert_{\Gamma_{-}}=-u_{y}\Big\vert_{\Gamma_{-}}, \qquad
\theta(t)=u\Big\vert_{\gamma}.
\end{align*} 
 Introduce the Fourier transforms of $\varphi$, $\psi$ by the formulas
\begin{equation*}
\tilde{\varphi}(p)=\ds\int e^{-ipx} \varphi(x)\;dx,\qquad \tilde{\psi}(p)=\ds\int e^{-ipx} \psi(x)\;dx.
\end{equation*}
Integrals without limits mean the integration over $\R$.\\
Denote ${\bf r}(t)=\Big(X(t), Y(t)\Big)$,~ ${\bf m}(t)=\Big(-\dot{Y}(t),\dot{X}(t)\Big)$.
Passing in (\ref{GF}) to the limits as $\xi,\eta\to \gamma\text{ or } \Gamma_\pm$, and passing to the Fourier transforms of $\varphi$, $\psi$, it can be shown that    $\tilde{\varphi}(p)$, $\tilde{\psi}(p)$,  and $\theta(t)$ satisfy the following system of integral equations:
\end{frame}
\begin{frame}
\begin{equation}\label{theta+}
\theta(t)+\ds\int\limits_{-\pi}^{\pi} M(t,s) \theta(s)\; ds=\ds\int M_1(t,p)\tilde{\varphi}(p)\;dp+\ds\int M_2(t,p)\tilde{\psi}(p)\;dp
\end{equation}
where
\begin{equation}\label{M1}
M(t,s)=-\ds\frac{\varepsilon k}{2}\;\ds\frac{N'_0\Big(\varepsilon k\abs{{\bf r}(s)-{\bf r}(t)}\Big)}{\abs{{\bf r}(s)-{\bf r}(t)}}\Big({\bf r}(s)-{\bf r}(t)\Big)\cdot {\bf m}(s),
\end{equation}
%$N_0(r)$ is the Neumann function, 
\begin{align}\label{KM1}
& M_1(t,p)=\ds\frac{1}{4\pi} e^{ip\varepsilon X}\Bigg(\ds\frac{1}{\tau(p)} e^{-\tau(p)h^{-}}+\ds\frac{1}{\check{\tau}(p)} e^{-\check{\tau}(p)h^{-}}\Bigg),\,\, h^{-}=b-a-\varepsilon Y(t)\\\nonumber\\\label{KM2}
& M_2(t,p)=\ds\frac{1}{4\pi} e^{ip\varepsilon X}\Bigg(\ds\frac{1}{\tau(p)} e^{-\tau(p)h^{+}}+\ds\frac{1}{\check{\tau}(p)} e^{-\check{\tau}(p)h^{+}}\Bigg),\,\, h^{+}=b+a+\varepsilon Y(t).
\end{align}

\end{frame}
\begin{frame}

\begin{equation}\label{tilde{var}}
\begin{array}{lll}
\tilde{\varphi}(p)\sinh \Big(2b\tau\Big)= \varepsilon \ds\int\limits_{-\pi}^{\pi} e^{-ip\varepsilon X}\Bigg(ip\dot{Y}\sinh \Big(h^{+} \tau\Big)+\tau\dot{X}\cosh\Big(h^{+} \tau\Big)\Bigg)\theta\;dt
\end{array}
\end{equation}

\begin{equation}\label{tilde{psi}}
\begin{array}{lll}
\tilde{\psi}(p)\sinh \Big(2b\tau\Big)= \varepsilon \ds\int\limits_{-\pi}^{\pi} e^{-ip\varepsilon X}\Bigg(ip\dot{Y}\sinh \Big(h^{-} \tau\Big)-\tau\dot{X} \cosh\Big(h^{-}\tau\Big)\Bigg)\theta\;dt;
\end{array}
\end{equation}
here the functions $\tau(p)$, $\check{\tau}(p)$ are the analytic continuations of $\big(p^2-k^2\big)^{1/2}$ to the complex plane $p\in \C$ with certain cuts coinciding with the arithmetical square root for $p>k$.
For real $p$, $\tau(p)=\check{\tau}(p)=\sqrt{p^2-k^2}$ for $\abs{p}>k$ and  $\tau(p)=-\check{\tau}(p)=-i\sqrt{k^2-p^2}$ for $\abs{p}<k$. 

\end{frame}
\begin{frame}
\begin{figure}[htbp]
\centering
\includegraphics[scale=0.27]{branch.pdf}
\end{figure}
\centerline{Cuts for $\tau(p)$}
\end{frame}
\begin{frame}
\begin{figure}[htbp]
\centering
\includegraphics[scale=0.27]{bra33.pdf}
\end{figure}
\centerline{Cuts for $\check{\tau}(p)$}
\end{frame}
\begin{frame}{Connection to the exterior Neumann problem}
Consider equation (\ref{theta+}). We   solve this equation with respect to $\theta$ and substitute its solution in (\ref{tilde{var}}), (\ref{tilde{psi}}) thus reducing our system to two equations for $\tilde\varphi$, $\tilde\psi$. Consider the kernel $M(t,s)$ (see (\ref{M1})). The function $N'_0(r)$ admits the following  expansion valid for small $r$:   
\begin{equation*}
N'_0(r)=\ds\frac{2}{\pi}\Bigg(\ds\frac{1}{r}-\ds\frac{r}{2}\ln\ds\frac{r}{2}+\ds\frac{r}{2}\Big(\ds\frac{1}{2}-\gamma\Big)\Bigg)+O(r^3\ln r),
\end{equation*}
where the $O$-symbol is analytic in $r$ and $r\ln r$.
Hence, 
\begin{equation}\label{Mser}
M(t,s)=M^{(0)}(t,s)+\varepsilon^2\ln\varepsilon\; M^{(1)}\big(t,s,\varepsilon,\varepsilon\ln\varepsilon\big)+\varepsilon^2\; M^{(2)}\big(t,s,\varepsilon,\varepsilon\ln\varepsilon\big).
\end{equation}
\end{frame}
\begin{frame}
Here
\begin{equation}\label{M1O}
M^{(0)}=-2\ds\frac{\partial G_0\Big({\bf r}(s)-{\bf r}(t)\Big)}{\partial n} \abs{{\bf m}(s)},
\end{equation}
with $G_0(x,y)=\ds\frac{1}{2\pi} \ln r$,~$r=\sqrt{x^2+y^2}$ and $\partial/\partial n$ being the derivative along the inward-looking normal to $\Gamma$; the normal derivative is calculated at the point $\Big(X(s), Y(s)\Big)$. The kernels $M^{(1,2)}$ are smooth in $t$, $s$ (at least of class $C^1$) and analytic in $\varepsilon$, $\varepsilon_1=\varepsilon\ln\varepsilon$, and hence the integral operators with these kernels are bounded in $C[-\pi,\pi]$.\\
Note that $G_0(x,y)$ is the fundamental solution of the Laplace equation and hence the operator $\hat M^{(0)}$ coincides with the integral operator for the exterior Neumann problem (\ref{Dlta}).
It is well-known  that the operator $1+\hat{M}^{(0)}$ is invertible in $C[-\pi,\pi]$, and hence the operator $1+\hat{M}$ also is. 
\end{frame}

\begin{frame}{Reduction to two equations}
Solving (\ref{theta+}) with respect to $\theta$ and substituting in (\ref{tilde{var}}), (\ref{tilde{psi}}), we obtain a system of two equations for $\tilde\varphi$ and $\tilde\psi$ with integral operators whose kernels are analytic in $p$. This system is solved by using the  scheme applied   to the Schr\"odinger equation and this yields our main  results since the equations for $\tilde\varphi$ and $\tilde\psi$ are {\it equivalent} to problem  (\ref{PD1}), (\ref{PD2}).
\end{frame}

\begin{frame}{Reference}

Zhevandrov, P., Merzon, A., Romero Rodr\'\i guez, M.I. et al. Discrete and Embedded Trapped Modes in a Plane Quantum Waveguide with a Small Obstacle: Exact Solutions. {\it Acta Appl Math} 196, 8 (2025). https://doi.org/10.1007/s10440-025-00720-2

%\centerline{ \textbf{Perturbed quantum waveguide}}



%\centerline{Nazarov S  \emph{Theor Math Phys}, \textbf{167}(2011), 606--627}

%\vspace{0.5cm}

%\centerline{ \textbf{Timoshenko system}}



%\centerline{Aya H, Cano R, Zhevandrov P  \emph{J Eng Math}, \textbf{77}(2012), 87--104}

%\vspace{0.5cm}

%\centerline{ \textbf{Two-layer shallow water}}



%\centerline{Romero Rodr\'\i guez M I, Zhevandrov P  \emph{J Fluid Mech}, \textbf{753}(2014), 427--447}

%\vspace{0.5cm}

%\centerline{\textbf{One-layer cylinder}}

%\centerline{Garibay F, Zhevandrov P \emph{Russ J Math Phys}, \textbf{22}(2015), 174--183}

\end{frame}
\end{document}
