% $Header: /cvsroot/latex-beamer/latex-beamer/solutions/generic-talks/generic-ornate-15min-45min.en.tex,v 1.5 2007/01/28 20:48:23 tantau Exp $

\documentclass{beamer}

% This file is a solution template for:

% - Giving a talk on some subject.
% - The talk is between 15min and 45min long.
% - Style is ornate.



% Copyright 2004 by Till Tantau <tantau@users.sourceforge.net>.
%
% In principle, this file can be redistributed and/or modified under
% the terms of the GNU Public License, version 2.
%
% However, this file is supposed to be a template to be modified
% for your own needs. For this reason, if you use this file as a
% template and not specifically distribute it as part of a another
% package/program, I grant the extra permission to freely copy and
% modify this file as you see fit and even to delete this copyright
% notice.


\mode<presentation>
{
  \usetheme{Madrid}
  % or ...
  \setbeamercovered{transparent}
  \setbeamertemplate{theorems}[numbered]
  \setbeamertemplate{figures}[numbered]
  % or whatever (possibly just delete it)
}


%\usepackage[spanish]{babel}
% or whatever
\usepackage{graphicx}
\usepackage[latin1]{inputenc}
% or whatever

\usepackage{times}
\usepackage[T1]{fontenc}
% Or whatever. Note that the encoding and the font should match. If T1
% does not look nice, try deleting the line with the fontenc.
\usepackage{amsmath,amsthm}
\usepackage{amssymb}
\usepackage{amsmath}

\usepackage{graphicx}
\usepackage{float}
\usepackage{amsmath}
\usepackage{amsthm}
\usepackage{amssymb}
\usepackage{amscd}
\usepackage{amsfonts}
\usepackage{makeidx}
\usepackage{enumerate}      
\usepackage{IEEEtrantools}
%\usepackage{cases}
%\usepackage{cases}
\usepackage{mathrsfs}

%\usepackage{amsmath,empheq}

\usepackage{color}

%%%%%%%%%%%%%%%%%%%%%%%%%%%%%%%%%%%%%%%%%%%%%%%
%\usepackage{showkeys}
\usepackage{amsmath}

\usepackage{graphicx}
\usepackage{float}
\usepackage{amsmath}
\usepackage{amsthm}
\usepackage{amssymb}
\usepackage{amscd}
\usepackage{amsfonts}
\usepackage{makeidx}
\usepackage{enumerate}      
\usepackage{IEEEtrantools}
%\usepackage{cases}
%\usepackage{cases}
\usepackage{mathrsfs}

%\usepackage{amsmath,empheq}


\usepackage[latin1]{inputenc}


%%%%%%%%%%%%%%%%%%%%%%%%%%%%%%%%%%%%%%%%%%%%%
\usepackage [all]{xy}
%%%%%%%%%%%%%%%%%%%%%%%%%%%%%%%%%%%%%%%%%%
\usepackage{cases}

\usepackage{amsmath}

\usepackage{cancel}

\usepackage{mathrsfs}



\newtheorem{teo}{Theorem}[section]
    \newtheorem{lem}[teo]{Lemma}
    \newtheorem{prop}[teo]{Proposition}
    \newtheorem{coro}[teo]{Corollary}
    \newtheorem{defn}[teo]{Definition}
    \newtheorem{obs}[teo]{Remark}
    \newtheorem{exa}[teo]{Example}
%\newtheorem{theorem}[section]{Teorema}
%\renewcommand{\thetheorem}{\arabic{section}.\arabic{theorem}}
%\renewcommand{\theequation}{\thesection.\arabic{equation}}  %%% PARA ENUMERAR LAS %ECUACIONES POR SECCIÓN CON NÚMEROS ARÁBIGOS
\newcommand{\re}{\ref}
\newcommand{\fr}{\frac}
\newcommand{\ci}{\cite}
\newcommand{\ov}{\overline}
    \newtheorem*{dem}{\textsc{Proof}}
    \newcommand{\bdem}{\begin{dem}}
    \newcommand{\edem}{\end{dem}}
%%%%%%%%%%%%%%
     \newcommand{\be}{\begin{equation}}
    \newcommand{\ee}{\end{equation}}
     \newcommand{\ba}{\begin{array}}
    \newcommand{\ea}{\end{array}}
\newcommand{\beqn}{\begin{eqnarray}}
    \newcommand{\eeqn}{\end{eqnarray}}
    \newcommand{\bl}{\begin{lem}}
    \newcommand{\el}{\end{lem}}
    \newcommand{\bp}{\begin{prop}}
    \newcommand{\ep}{\end{prop}}
\newcommand{\ds}{\displaystyle}
     \newcommand{\la}{\label}
     \newcommand{\al}{\alpha}
      \newcommand{\Ga}{\Gamma}
      \newcommand{\cC}{\cal C}
    \newcommand{\De}{\Delta}
     \newcommand{\om}{\omega}
    \newcommand{\vp}{\varphi} \newcommand{\ga}{\gamma}
    \newcommand{\R}{\mathbb{R}}
    \newcommand{\C}{\mathbb{C}}
     \newcommand{\Z}{\mathbb{Z}}
     \newcommand{\no}{\noindent}
     \newcommand{\Res}{{\rm Res}}
     \newcommand{\vare}{\varepsilon}
   
   
   %%%%%%%%%%%%%%%%%%%%%%%%%%%%%%%%%%%%%%%%%%%%%%%%%%%%%%%%%%%%%% 
   \providecommand{\abs}[1]{\lvert#1\rvert}
\providecommand{\norm}[1]{\lVert#1\rVert}
 
    
    
   %%%%%%%%%%%%%%%%%%%%%%%%%%%%%%%%%%%%%%%%%%%%%%%%%%%%%%%%%%% 
    
    
    \IEEEyessubnumber

\title[Trapped Modes]{Discrete and embedded trapped modes in a quantum waveguide with a small obstacle}
\author[Zhevandrov et al.]{
  P. Zhevandrov, A.  Merzon, M.I. Romero Rodr\'iguez
  }
\institute[UMSNH, UMNG]{
{Facultad de Ciencias F\'{\i}sico-Matem\'aticas, UMSNH, Morelia, Mexico}\\
{Instituto de  F\'{\i}sica y Matem\'aticas, UMSNH, Morelia, Mexico}\\
{Universidad Militar ``Nueva Granada'', Bogot\'a, Colombia}
}
\date[2025]{XIII Taller de Geometr\'\i a y Sistemas Din\'amicos}


\subject{Talks}
% This is only inserted into the PDF information catalog. Can be left
% out.



% If you have a file called "university-logo-filename.xxx", where xxx
% is a graphic format that can be processed by latex or pdflatex,
% resp., then you can add a logo as follows:

%\pgfdeclareimage[height=1 cm]{university-logo}{university-logo-filename}
%\logo{\pgfuseimage{university-logo}}



% Delete this, if you do not want the table of contents to pop up at
% the beginning of each subsection:
%\AtBeginSubsection[]
%{
% \begin{frame}<beamer>{Outline}
%    \tableofcontents[currentsection,currentsubsection]
%  \end{frame}
%}


% If you wish to uncover everything in a step-wise fashion, uncomment
% the following command:

%\beamerdefaultoverlayspecification{<+->}


\begin{document}
\begin{frame}
  \titlepage
\end{frame}

\begin{frame}{Outline}
  \tableofcontents
  % You might wish to add the option [pausesections]
\end{frame}




% Since this a solution template for a generic talk, very little can
% be said about how it should be structured. However, the talk length
% of between 15min and 45min and the theme suggest that you stick to
% the following rules:

% - Exactly two or three sections (other than the summary).
% - At *most* three subsections per section.
% - Talk about 30s to 2min per frame. So there should be between about
% 15 and 30 frames, all told.
\section{Numerical methods}
\begin{frame}{Numerical method}
  The Helmholtz equation was solved numerically using a simple version of the Boundary Element Method (Linton et. al, 1992). The solution $u$ on a point $P \in \Omega$ has the integral representation
  \begin{equation}
    \label{eq:1}
    u(P) = \int_{\gamma} \left[ u(q)\, 
      \frac{\partial}{\partial n} G(P,q) 
- G(P,q)\, \frac{\partial u}{\partial n}(q) \right] \, ds ,
  \end{equation}
  where $q$ is a point over the obstacle boundary $\gamma$, and $G(P,q)$ is the Green's function of the Helmholtz equation that satisfies the homogeneous Dirichlet boundary conditions $G(P,q) = 0$ for $P \in \Gamma_{\pm}$. For a point $p \in \gamma$, the integral representation becomes
  \begin{equation}
    \label{eq:1}
    \frac{1}{2}u(p) = \int_{\gamma} \left[ u(q)\, 
      \frac{\partial}{\partial n} G(p,q) 
- G(p,q)\, \frac{\partial u}{\partial n}(q) \right] \, ds.
  \end{equation}
  
\end{frame}


\begin{frame}
  If $\gamma$ is parametrized in polar coordinates, radius $\rho = \rho(\theta), 0 \le \theta < 2\pi$, the integral representation becomes
  \begin{equation}
    \label{eq:2}
      \frac{1}{2}u(\psi) = \int_{0}^{2\pi} 
u(\theta) \, \frac{\partial}{\partial n} 
G(\psi,\theta)\, w(\theta) \, d\theta, 
  \end{equation}
  where $w(\theta) = [\rho^2(\theta) + \rho'^2(\theta)]^{1/2}$. Discretizing $\theta$, $\theta_i = (i + \frac{1}{2})\Delta\theta$, $\Delta\theta = 2\pi/M$, $i=1, 2,\ldots$, the integral equation becomes a homogeneous linear system
  \begin{equation}
    \label{eq:3}
    \sum_{j=1}^M A_{ij} u_j = 0,~~i=1,2,\ldots,M.
  \end{equation}
  where the unknowns are $u_j = u(\theta_j)$ and the matrix elements $A_{ij}$ depend on $\frac{\partial G}{\partial n}(\psi_i, \theta_j)$, which, in turn, depend on the assumed value of the wave number $k$. $G(\psi,\theta)$ is computed by the method of the lattice sums (Linton, 1998).
\end{frame}

\begin{frame}{Results}
  The system has nontrivial solutions $[u_1, u_2, \ldots, u_M]$ (trapped modes) for values $k_{num}$ of $k$ such that $\det A = 0$.

  \begin{columns}
    \begin{column}{0.3\textwidth}
      For a circle of radius $R=1$, $\mu=R^2$, $b=1$, $a_0^* = 0.295$, $a = 0.6$, $\varepsilon=10^{-2}$,
      \begin{align}
        \sigma_{num} &= \sqrt{\Lambda_1 - k_{num}^2} \nonumber\\
                     &= 3.7341 \times 10^{-4} \nonumber
      \end{align}
    \end{column}
    
    \begin{column}{0.7\textwidth}
      \begin{figure}[htbp]
        \centering
        \includegraphics[scale=0.4]{determinant.pdf}
        % \caption{Small obstacle in a waveguide}\label{Gam_d}
      \end{figure}
      
    \end{column}
  \end{columns}
  
\end{frame}

\begin{frame}
  Comparison of $\sigma_{num}$ with the analytical $\sigma$ as functions of $\varepsilon$, for a circle with $R=1$ and $a = 0.6$.
  \begin{figure}[htbp]
    \centering
    \includegraphics[scale=0.50]{sigma_analytic_numeric.pdf}
    % \caption{Small obstacle in a waveguide}\label{Gam_d}
  \end{figure}

\end{frame}

\begin{frame}
  \centerline{Difference between $\sigma_{num}$ and the analytical $\sigma$}
  \begin{figure}[htbp]
    \centering
    \includegraphics[scale=0.50]{difference_sigma.pdf}
    % \caption{Small obstacle in a waveguide}\label{Gam_d}
  \end{figure}

\end{frame}



\begin{frame}{References}
  \begin{itemize}
  \item Zhevandrov, P., Merzon, A., Romero Rodr\'\i guez, M.I. et al. Discrete and Embedded Trapped Modes in a Plane Quantum Waveguide with a Small Obstacle: Exact Solutions. {\it Acta Appl Math} 196, 8 (2025). https://doi.org/10.1007/s10440-025-00720-2

  \item Linton, C. M., and D. V. Evans. "Integral equations for a class of problems concerning obstacles in waveguides." Journal of Fluid Mechanics 245 (1992): 349-365.

  \item Linton, C. M. "The Green's function for the two-dimensional Helmholtz equation in periodic domains." Journal of Engineering Mathematics 33.4 (1998): 377-401.
    
  \end{itemize}

%\centerline{ \textbf{Perturbed quantum waveguide}}



%\centerline{Nazarov S  \emph{Theor Math Phys}, \textbf{167}(2011), 606--627}

%\vspace{0.5cm}

%\centerline{ \textbf{Timoshenko system}}



%\centerline{Aya H, Cano R, Zhevandrov P  \emph{J Eng Math}, \textbf{77}(2012), 87--104}

%\vspace{0.5cm}

%\centerline{ \textbf{Two-layer shallow water}}



%\centerline{Romero Rodr\'\i guez M I, Zhevandrov P  \emph{J Fluid Mech}, \textbf{753}(2014), 427--447}

%\vspace{0.5cm}

%\centerline{\textbf{One-layer cylinder}}

%\centerline{Garibay F, Zhevandrov P \emph{Russ J Math Phys}, \textbf{22}(2015), 174--183}

\end{frame}
\end{document}
