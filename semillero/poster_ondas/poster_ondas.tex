%==============================================================================
%== template for LATEX poster =================================================
%==============================================================================
%
%--A0 beamer slide-------------------------------------------------------------
\documentclass[final]{beamer}
\usepackage[orientation=portrait,size=a0,
            scale=1.25         % font scale factor
           ]{beamerposter}
           
\geometry{
  hmargin=2.5cm, % little modification of margins
}

\usepackage{amsmath}%,stmaryrd}
\usepackage{graphicx}
\usepackage[percent]{overpic}
\usepackage[spanish]{babel}

\usepackage[utf8]{inputenc}
\usepackage{color}
\usepackage{array}
\def\cred#1{{\color{red}#1}}
\linespread{1.15}
%
%==The poster style============================================================
\usetheme{sharelatex}

%==Title, date and authors of the poster=======================================
\title
%[\email{\underline{paula.silva@correo.usa.edu.co}} \  \email{alejandro.garzon@usa.edu.co} \ \email{jucordov@uniandes.edu.co} \ \email{miguela.solano@correo.usa.edu.co}] % Conference
{ % Poster title
Análisis matemático de las propiedades acústicas y electromagnéticas de materiales en minas antipersonales
}

\author{ % Authors
  \underline{Karen Lorena Pérez}\inst{1},
  \underline{Javier Alejandro Castillo}\inst{1},
  \underline{Kevin Pineda Alonso}\inst{1},
  María Isabel Romero \inst{1},
  Alejandro Garzón \inst{1},
  Laura Manuela Orjuela \inst{1},
}
\institute
[Very Large University] % General University
{
\inst{1} Universidad Militar Nueva Granada, % \\[0.3ex]
%Bogota D.C.,
%Colombia
}

\date{\today}



\begin{document}
\begin{frame}[t]
%==============================================================================

\begin{multicols}{3}
%==============================================================================
%==The poster content==========================================================
%==============================================================================
\section{Resumen}

\section{Introducción}
\section{Ecuación de onda en una dimensión}
Como primera aproximación a la dispersión de ondas por parte de obstáculos estudiaremos la ecuación de onda en una dimensión. Sea $u$ el desplazamiento transversal de una cuerda (o también el campo de presión de ondas sonoras planas) que evoluciona de acuerdo con
\begin{equation}
  \label{eq:1}
  \frac{\partial ^2u}{\partial t^2} = c^2 \frac{\partial^2 u}{\partial x^2},
\end{equation}
donde $c = T/\mu$ es la velocidad de las ondas. Para una cuerda infinita tenemos solutiones de la forma
\begin{equation}
  \label{eq:2}
  u(x,t) = p(x - ct) + q(x + ct).
\end{equation}
Si la densidad $\mu$ de la cuerda cambia en un punto, allí ocurre reflexión y transmisión de ondas
\begin{align}
  \label{eq:3}
  u_1(x,t) &= R_1(x - c_1 t) + L_1(x + c_1t) \\
  u_2(x,t) &= L_2(x + c_2 t)
\end{align}
Por continuidad de la solución en la interface \cite{smith2010waves},
\begin{align}
  \label{eq:4}
  &\left[L_1 = \frac{c_2 - c_1}{c_2 + c_1} R_1\right]_{x=0} \\
  &\left[R_2 = \frac{2c_2}{c_2 + c_1} R_1 \right]_{x=0}.
\end{align}
La solución se muestra en la Fig. \ref{fig:reflex}.
\vskip1ex
\begin{figure}
%\centering
  \includegraphics[width=1.0\columnwidth]{reflex_trans_1.pdf}
  \caption{Reflexión y transmisión de ondas}
  \label{fig:reflex}
\end{figure}
\vskip2ex

%% \vskip1ex
%% \begin{figure}
%% \centering
%% %\includegraphics[width=0.5\columnwidth]{fusion_evaporation_tall}
%% \setlength{\unitlength}{1cm}
%% \begin{picture}(20,20)
%% \put(10,10){\circle{1}}
%% \end{picture}
%% \caption{Reacci\'on de fusi\'on-evaporaci\'on.}
%% \end{figure}
%% \vskip2ex

\section{Solución analítica para una cuerda finita}
Para una cuerda finita $0 \le x \le L$, la ecuación de onda se resuelve por separación de variables
\begin{equation}
  \label{eq:5}
  u(x,t) = v(x)w(t).
\end{equation}
Remplazando \cite{olver2014introduction},
\begin{align}
  v'' - \frac{\lambda}{c^2} v &= 0 \\
  w'' - \lambda w &= 0.
\end{align}
Para condiciones de frontera de Dirichlet homogéneas, $u(0,t) = u(L,t) = 0$,
\begin{equation}
  \label{eq:7}
  v_n(x) = \sin\left(\frac{\sqrt{\lambda_n}}{c}x\right)
\end{equation}
\begin{equation}
  \label{eq:8}
  w_n(t) = \cos(\sqrt{\lambda_n} t),~~\tilde{w}_n(t) = \sin(\sqrt{\lambda_n} t)
\end{equation}
\begin{equation}
  \label{eq:9}
  \lambda_n = -\left(\frac{n\pi c}{L}\right)^2,~~n = 1,2, \ldots  
\end{equation}
La soluci\'on general es una superposición, conocida como una serie de Fourier,
\begin{align}
  \label{eq:10}
  u(x,t) &= \sum_{n=1}^\infty \left[a_n w_n(t) +  b_n \tilde{w}_n(t) \right]v_n(x) \\
         &= \sum_{n=1}^\infty \left[a_n\cos\left(\frac{n\pi ct}{L}\right) +  b_n\sin\left(\frac{n\pi ct}{L}\right)\right]\sin\left(\frac{n\pi x}{L}\right) \label{eq:fourier}
\end{align}
Los coeficientes $a_n$ y $b_n$ dependen de la condición inicial $u(x,0) = f(x)$ y $u_t(x,0) = g(x)$. La Fig. \ref{fig:triangulo} muestra la solución analítica \eqref{eq:fourier} para posición inicial $f(x)$ en forma de tienda y velocidad inicial $g(x)=0$.
\vskip1ex
\begin{figure}
%\centering
  \includegraphics[width=1.0\columnwidth]{onda_triangulo.pdf}
  \caption{Solución expresada como serie de Fourier}
  \label{fig:triangulo}
\end{figure}
\vskip2ex


\section{Solución numérica para una cuerda finita}
En el método de diferencias finitas, la solución se aproxima en una malla de puntos discretos, $(x_m, t_j), x_m = m\Delta x, m=0, 1, \ldots, N, t_j = j\Delta t, j=0, 1, \ldots$.
La segunda derivada $\partial^2 u/\partial x^2$ en \eqref{eq:1} puede aproximarse mediante
\begin{equation}
  \label{eq:11}
  \frac{\partial^2 u}{\partial x^2} = \frac{u_{m+1,j} - 2u_{m,j} + u_{m-1,j}}{(\Delta x)^2} + O[(\Delta x)^2].
\end{equation}
Replazando en \eqref{eq:1}, con una aproximación análoga para $\partial^2 u/\partial t^2$ se obtiene

\begin{subequations}
  \label{eq:num}
\begin{align}
  \label{eq:6}
  {\bf u}^{(0)} &= {\bf f} \\
  {\bf u}^{(1)} &= \frac{1}{2}B{\bf u}^{(0)} + {\bf g}\Delta t + \frac{1}{2}{\bf b}^{(0)} \\
  {\bf u}^{(j+1)} &= B{\bf u}^{(j)} - {\bf u}^{(j-1)} + {\bf b}^{(j)},
\end{align}
\end{subequations}
donde
\begin{equation}
  \label{eq:11}
  B =
  \begin{bmatrix}
    2(1-\sigma^2) & \sigma^2      &                      \\
    \sigma^2      & 2(1-\sigma^2) & \sigma^2             \\
                  & \sigma^2      & \ddots     & \ddots  \\
                  &               & \ddots     & \sigma^2 \\
                  &               & \sigma^2   & 2(1-\sigma^2)
  \end{bmatrix},
  {\bf u}^{(j)} =
  \begin{bmatrix}
    u_{1,j} \\
    u_{2,j} \\
    \vdots \\
    u_{N-2,j} \\
    u_{N-1,j}
  \end{bmatrix},
\end{equation}
con
\begin{equation}
  \label{eq:12}
  \sigma = c\Delta t/\Delta x.
\end{equation}
 La figura \ref{fig:numerica} muestra la solución numérica calculada mediante las ecuaciones \eqref{eq:num} para una posición inicial $f(x)$ gaussiana y velocidad inicial $g(x)=0$.
\vskip1ex
\begin{figure}
%\centering
  \includegraphics[width=1.0\columnwidth]{onda_numerico.pdf}
  \caption{Solución numérica de la ecuación de onda}
  \label{fig:numerica}
\end{figure}
\vskip2ex



\section{Conclusiones}

%==============================================================================
%==End of content==============================================================
%==============================================================================

%--References------------------------------------------------------------------

\subsection{Referencias}

\bibliographystyle{unsrt}
\bibliography{references}


% \begin{thebibliography}{9}
% \bibitem{garzon2020} A. Garzón, W. Rodríguez, F. Cristancho, and M. Tao, {\em AhKin: a modular and efficient code for the Doppler Shift Attenuation Method}, Computer Physics Communications, Vol. 246 , Pag. 106854, 2020 (DOI: 10.1016/j.cpc.2019.07.017).

% \bibitem{rodriguez2019} W. Rodriguez, F. Cristancho, S. L. Tabor, A. Kardan, I. Ragnarsson, R. A. Haring-Kaye, J. Döring, D. G. Sarantites, and A. Garzón, {\em High spin states of the normally deformed bands of $^{83}$Y}, Physical Review C, Vol. 100, Pag. 024327, 2019 (DOI: 10.1103/PhysRevC.100.024327).
% \end{thebibliography}




%--End of references-----------------------------------------------------------


\end{multicols}

%==============================================================================
\end{frame}
\end{document}
