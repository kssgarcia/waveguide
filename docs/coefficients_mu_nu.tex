\documentclass[11pt]{article}
% \setlength{\voffset}{-2cm}
% \setlength{\hoffset}{-2cm}
% \setlength{\textwidth}{16cm}
% \setlength{\textheight}{25cm}
% \setlength{\marginparwidth}{1cm}
\usepackage[english,spanish,activeacute]{babel}
\usepackage{graphicx,array}
\usepackage{amsmath}
%\usepackage{pstricks}
\usepackage{multicol}
\usepackage{color}
\usepackage{listings}
\usepackage[inline]{enumitem}
\usepackage[colorlinks]{hyperref}
% \usepackage[hypertexnames=false]{hyperref}
% \pagestyle{empty}
\decimalpoint

\definecolor{darkgray}{rgb}{0.66, 0.66, 0.66}
\definecolor{darkorange}{rgb}{1.0, 0.55, 0.0}
\definecolor{gray}{rgb}{0.97,0.97,0.99}
\definecolor{teal}{rgb}{0.0, 0.5, 0.5}
\definecolor{comment}{rgb}{0.6, 0, 0.9}
\lstdefinestyle{mystyle}{
	language = Python,
	backgroundcolor=\color{gray},
	commentstyle=\color{comment},
	keywordstyle=\bfseries\color{darkorange},
	numberstyle=\scriptsize\color{darkgray},
	stringstyle=\color{teal},
	basicstyle=\scriptsize\ttfamily,%\linespread{1}
	breakatwhitespace=false,
	breaklines=true,
	captionpos=t,
	keepspaces=false,
	numbers=left,
	numbersep=3pt,
	showspaces=false,
	showstringspaces=false,
	showtabs=false,
	tabsize=4
}
\lstset{style=mystyle}

\synctex=1

\begin{document}
\begin{center}
{\Large Dipole moments}
\end{center}
This document derives the equations (3.15) of the paper {\tt ICOVP27.06.25PZh.pdf}. The document follows the notation and conventions of many textbooks, which are different from those in the paper. In particular, the disturbance potential denoted by $\Psi$ in the paper is denoted here by $\phi_d$ and the uniform undisturbed fluid flow, which in paper points in the negative $y$ direction, here points in the positive $x$ direction. We begin by defining the physical problem where the disturbance potential arises.

\section{The physical problem}
Let us determine how a rigid cylinder disturbs an otherwise uniform fluid flow. Suppose the cylinder axis is perpendicular to the $(x,y)$ plane and has an arbitrary cross section described by a closed contour $\Gamma$. We want to find the two-dimensional velocity field ${\bf V}=(u,v)$ of the irrotational flow,
\begin{equation}
  \label{eq:19}
  \nabla \times {\bf V} = {\bf 0},
\end{equation}
of an incompressible fluid,
\begin{equation}
  \label{eq:33}
  \nabla \cdot {\bf V} = 0,
\end{equation}
around the closed contour $\Gamma$. The fluid can slide along the contour, but not enter it. Therefore, the component of the velocity perpendicular to the contour must be zero,
\begin{equation}
  \label{eq:36}
  {\bf V}\cdot{\bf n} = 0,~\mathrm{over}~\Gamma.
\end{equation}
From \eqref{eq:19} it follows that there must exist a scalar field $\phi$ (known as the velocity potential) such that
\begin{equation}
  \label{eq:37}
  {\bf V} = \nabla\phi.
\end{equation}
Substituting \eqref{eq:37} into \eqref{eq:33}, we get the Laplace equation
\begin{equation}
  \label{eq:38}
  \nabla^2\phi = 0.
\end{equation}
Moreover, using \eqref{eq:37}, \eqref{eq:36} takes the form,
\begin{equation}
  \label{eq:42}
  \nabla\phi\cdot{\bf n} = \frac{\partial\phi}{\partial n} = 0,~\mathrm{over}~\Gamma.
\end{equation}
Far away from the cylinder, the velocity must approach the undisturbed uniform flow with magnitude $U$ and direction ${\bf\hat{x}}$,
\begin{equation}
  \label{eq:43}
\lim_{r\rightarrow\infty} \nabla\phi = U{\bf\hat{x}},
\end{equation}
$r = \sqrt{x^2 + y^2}$.
Equations \eqref{eq:38}, \eqref{eq:42}, and \eqref{eq:43} determine $\phi$ up to an anbitrary additive constant.

Because of the condition \eqref{eq:43}, $\phi$ can be written as
\begin{equation}
  \label{eq:45}
  \phi(x,y) = U x + \phi_d(x,y),
\end{equation}
where $U x$ is the potential of the undisturbed flow and $\phi_d(x,y)$ is the potential introduced by the presence of the cylinder (disturbance potential), which must satisfy
\begin{equation}
  \label{eq:47}
  \boxed{
    \lim_{r\rightarrow\infty} \nabla\phi_d = {\bf 0}.
    }
\end{equation}
Note that
\begin{align}
  \lim_{r\rightarrow\infty} \nabla\phi &= \lim_{r\rightarrow\infty} \nabla\left[U x + \phi_d(x,y)\right] \\
                                       &= U {\bf\hat{x}} + \lim_{r\rightarrow\infty} \nabla\phi_d(x,y) \\
                                       &= U {\bf\hat{x}} + {\bf 0} \\
                                       &= U {\bf\hat{x}}.
\end{align}
Additionally on the boundary of the cylinder, Eq. \eqref{eq:42} implies
\begin{align}
  \label{eq:48}
  \nabla\phi\cdot{\bf n} &= U{\bf\hat{x}}\cdot{\bf n} + \nabla\phi_d\cdot{\bf n} \\
                         &= U n_1 + \frac{\partial\phi_d}{\partial n} \\
                         &= 0,
\end{align}
where ${\bf n} = (n_1, n_2)$. Solving $\frac{\partial\phi_d}{\partial n}$ from the previous equation
\begin{equation}
  \label{eq:49}
  \boxed{
    \frac{\partial\phi_d}{\partial n} = -U n_1.
    }
  \end{equation}
  Since $\phi$ and $U x$ are solutions of the Laplace equation, $\phi_d$ must be one too,
  \begin{equation}
    \label{eq:50}
    \boxed{
      \nabla^2 \phi_d = 0.
      }
  \end{equation}
Equations \eqref{eq:47}, \eqref{eq:49}, and \eqref{eq:50} correspond to equations (3.13) in the paper.

\section{Laurent series expansion of the complex potential}
We can define a complex potential $F(z)$, $z=x + iy$, whose real part is the velocity potential $\phi$,
\begin{equation}
  \label{eq:51}
  F(z) = \phi(x,y) + i \psi(x,y).
\end{equation}
and which is analytical in the region outside the contour $\Gamma$. Since $F(z)$ is analytical, the Cauchy-Riemann relations imply that its imaginary part $\psi$ must be the {\it stream function}, which satisfies
\begin{equation}
  \label{eq:1}
  {\bf V} = \nabla\times \psi{\bf\hat{z}}
\end{equation}
and is constant over the flow streamlines, in particular over the contour $\Gamma$.

Due to \eqref{eq:45} and \eqref{eq:47}, the only term with a positive power of $z$ in the Laurent series expansion of $F(z)$ is that of degree one. Hence, the series has the form
\begin{equation}
  \label{eq:2}
  F(z) = U z + \frac{M_1}{z} + \frac{M_2}{z^2} + ...
\end{equation}
where, according to the theorem on the existence the Laurent series \cite{brown2014}, the complex coefficients $M_n = A_n + iB_n$ are given by
\begin{equation}
  \label{eq:3}
  M_n = \frac{1}{2\pi i}\oint_\Gamma F(z)z^{n-1}dz.
\end{equation}

The real part of equation \eqref{eq:2} is
\begin{equation}
  \label{eq:4}
  \phi(x,y) = U x + \frac{A_1 x}{r^2} + \frac{B_1 y}{r^2} + O\left(\frac{1}{r^2}\right).
\end{equation}
So
\begin{equation}
  \label{eq:4}
  \phi_d(x,y) = \frac{A_1 x}{r^2} + \frac{B_1 y}{r^2} + O\left(\frac{1}{r^2}\right),
\end{equation}
which corresponds to Eq. (3.14) in the paper.

For the coefficient $M_1$, \eqref{eq:3} takes the form
\begin{align}
  \label{eq:5}
  M_1 &= \frac{1}{2\pi i} \oint_\Gamma F(z) dz \\
      &=\frac{1}{2\pi i} \oint_\Gamma \left[\phi(x,y) + i\psi(x,y)\right]dz \\
      &=\frac{1}{2\pi i} \left[\oint_\Gamma \phi(x,y)dz + i\oint_\Gamma \psi(x,y)dz \right].\label{eq:phi_psi}
\end{align}
The stream function $\psi(x,y)$ is constant over $\Gamma$ and can chosen to be zero there, cancelling the second integral in \eqref{eq:phi_psi}. As a result,
\begin{align}
  \label{eq:7}
  M_1 &= \frac{1}{2\pi i}\oint_\Gamma \phi(x,y)(dx + i dy)\\
  &= \frac{1}{2\pi}\left[\oint_\Gamma \phi(x,y) dy -
    i\oint_\Gamma \phi(x,y) dx \right].\label{eq:phi_xy}
\end{align}
Replacing \eqref{eq:45} into \eqref{eq:phi_xy},
\begin{align}
  M_1 &= \frac{1}{2\pi}\left(\oint_\Gamma [U x + \phi_d(x,y)]dy -
        i\oint_\Gamma [U x + \phi_d(x,y)]dx \right) \\
  &= \frac{1}{2\pi}\left(U\oint_\Gamma  x dy + \oint_\Gamma  \phi_d(x,y) dy -i U\oint_\Gamma  x dx -i \oint_\Gamma \phi_d(x,y) dx \right).\label{eq:M1}
\end{align}
The first ingregral in \eqref{eq:M1} is equal to the area $S$ of the region enclosed by the contour $\Gamma$,
\begin{equation}
  \label{eq:10}
  \oint_\Gamma xdy = S,
\end{equation}
and the third integral vanishes,
\begin{equation}
  \label{eq:8}
  \oint_\Gamma x dx = 0.
\end{equation}
Therefore the real and imaginary parts of $M_1$ are given by
\begin{equation}
  \label{eq:9}
  A_1 = \frac{1}{2\pi}\left(U S + \oint_\Gamma \phi_d(x,y)dy \right)
  \end{equation}
  and
  \begin{equation}
    \label{eq:11}
    B_1 = -\frac{1}{2\pi}\oint_\Gamma \phi_d(x,y)dx.
  \end{equation}
  where it must be understood that if $z(t) = x(t) + i y(t)$ is a given parametrization of $\Gamma$, the elements of integration are
    \begin{align}
    \label{eq:12}
    dx &= \dot{x}(t) dt \\
    dy &= \dot{y}(t) dt.
    \end{align}
    The integrals can be expressed in terms of the components of the normal unit vector ${\bf n}$ that points inwadly and the arclenth element $dl$ in the following manner. The unit vector ${\bf T}$ tangent to $\Gamma$ is
    \begin{align}
      \label{eq:13}
      {\bf T} &= (T_1, T_2) \\
              &=  \left( \frac{\dot{x}(t)}{|\dot{z}(t)|}, \frac{\dot{y}(t)}{|\dot{z}(t)|} \right).
    \end{align}
    where $|\dot{z}(t)| = \sqrt{[\dot{x}(t)]^2 + [\dot{y}(t)]^2}$. The element of arclenth $dl$ is
  \begin{equation}
    \label{eq:6}
    dl = |\dot{z}(t)| dt.
  \end{equation}
  Therefore,
  \begin{align}
    dx &= T_1 dl \\
    dy &= T_2 dl.
  \end{align}
  The vector ${\bf n}$ is pependicular to ${\bf T}$, ${\bf n} = {\bf\hat{k}} \times {\bf T}$,
  \begin{align}
    {\bf n} &= (n_1, n_2) \\
            &= (-T_2, T_1).
  \end{align}
  As a result


  \begin{equation}
    \label{eq:9}
    \boxed{
      A_1 = \frac{1}{2\pi}\left(U S - \oint_\Gamma \phi_d(x,y)n_1 dl \right)
      }
  \end{equation}
  and
  \begin{equation}
    \label{eq:11}
    \boxed{
      B_1 = -\frac{1}{2\pi}\oint_\Gamma \phi_d(x,y) n_2 dl,
      }
    \end{equation}
 which correspond to equations (3.15) in the paper.
\begin{thebibliography}{99}
\bibitem{brown2014}J.W. Brown and R.V. Churchill, {\it Complex Variables and Applications}, Ninth Edition, McGraw-Hill, 2014.
\end{thebibliography}




 \end{document}

  