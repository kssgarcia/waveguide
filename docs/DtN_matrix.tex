\documentclass[11pt]{article}
% \setlength{\voffset}{-2cm}
% \setlength{\hoffset}{-2cm}
% \setlength{\textwidth}{16cm}
% \setlength{\textheight}{25cm}
% \setlength{\marginparwidth}{1cm}
% \usepackage[english,spanish,activeacute]{babel}
\usepackage{graphicx,array}
\usepackage{amsmath, amssymb}
%\usepackage{pstricks}
\usepackage{multicol}
\usepackage{color}
\usepackage{listings}
\usepackage[inline]{enumitem}
\usepackage[colorlinks]{hyperref}
% \usepackage[hypertexnames=false]{hyperref}
% \pagestyle{empty}
% \decimalpoint

\definecolor{darkgray}{rgb}{0.66, 0.66, 0.66}
\definecolor{darkorange}{rgb}{1.0, 0.55, 0.0}
\definecolor{gray}{rgb}{0.97,0.97,0.99}
\definecolor{teal}{rgb}{0.0, 0.5, 0.5}
\definecolor{comment}{rgb}{0.6, 0, 0.9}
\lstdefinestyle{mystyle}{
	language = Python,
	backgroundcolor=\color{gray},
	commentstyle=\color{comment},
	keywordstyle=\bfseries\color{darkorange},
	numberstyle=\scriptsize\color{darkgray},
	stringstyle=\color{teal},
	basicstyle=\scriptsize\ttfamily,%\linespread{1}
	breakatwhitespace=false,
	breaklines=true,
	captionpos=t,
	keepspaces=false,
	numbers=left,
	numbersep=3pt,
	showspaces=false,
	showstringspaces=false,
	showtabs=false,
	tabsize=4
}
\lstset{style=mystyle}

\synctex=1

\newcommand{\wt}[1]{\widetilde{#1}}
\newcommand{\ol}[1]{\overline{#1}}

\begin{document}
\begin{center}
{\Large Dirichlet-to-Neumannn operator term in matrix form}
\end{center}
We need to present in matrix form the following expression:
\begin{equation}
  \label{eq:1}
  \int_{\Sigma_L}\Lambda_k^+(u) v dy
\end{equation}
where $\Lambda_k^+$ is the Dirichlet-to-Neumann (DtN) operator on the right boundary of the domain $\Sigma_L = \{(x,y)|x=L\}$. Truncated to $N$ terms, the DtN operator has the form
\begin{equation}
  \label{eq:2}
  [\Lambda_k^+(u)](y) = \sum_{n=1}^N I \beta_n \langle u, \varphi_n \rangle \varphi_n(y),
\end{equation}
where $I$ denotes imaginary unit, $I = \sqrt{-1}$ (to be able to use $i$ as an index),
\begin{equation}
  \label{eq:3}
  \beta_n = \sqrt{k^2 - \lambda_n^2},~~\lambda_n = \frac{n\pi}{H},~~n=1,2,\ldots,
\end{equation}
\begin{equation}
  \label{eq:4}
  \varphi_n(y) = \sqrt{\frac{2}{H}} \sin(\lambda_n y),
\end{equation}
\begin{equation}
  \label{eq:5}
  \langle u, \varphi_n \rangle = \int_0^H u(L,y) \varphi_n(y) dy.
\end{equation}
Let us assume the grid has $M$ nodes and let $\phi_i(x,y)$, $i=1,2, \ldots, M$, denote the node funtions. We call $\Gamma^+$ the set of the indices of the nodes that lie on $\Sigma_L$. In terms of the $\phi_i(x,y)$, the functions $u$, $v$, restricted to $\Sigma_L$, and the image given to $u$ by the DtN operator have the forms
\begin{equation}
  \label{eq:8}
  u(L,y) = \sum_{i \in \Gamma^+}^M u_i\phi_i(L,y)
\end{equation}
\begin{equation}
  \label{eq:9}
  v(L,y) = \sum_{i \in \Gamma^+} v_i\phi_i(L,y)
\end{equation}
and
\begin{equation}
  \label{eq:6}
  [\Lambda_k^+(u)](y) = \sum_{i \in \Gamma^+} a_i \phi_i(L,y),
\end{equation}
where
\begin{equation}
  \label{eq:7}
  a_i = [\Lambda_k^+(u)](y_i),~~i \in \Gamma^+.
\end{equation}
Substituting \eqref{eq:6} and \eqref{eq:9} into \eqref{eq:1},
\begin{align}
  \label{eq:10}
  \int_{\Sigma_L}\Lambda_k^+(u) v dy &=
                                       \sum_{i\in\Gamma^+}\sum_{j\in\Gamma^+}  v_i a_j \int_0^H \phi_i(L,y)\phi_j(L,y)dy \\
  &= \sum_{i\in\Gamma^+}\sum_{j\in\Gamma^+} v_i C_{ij}^+ a_j \label{eq:10b}
\end{align}
where 
\begin{equation}
  \label{eq:12}
  C_{ij}^+ = \int_0^H \phi_i(L,y)\phi_j(L,y)dy.
\end{equation}
To arrive at the desired matrix form, we need to express the $a_i$ in terms of the $u_i$. We begin by replacing \eqref{eq:8} into \eqref{eq:5},
\begin{align}
  \label{eq:11}
  \langle u, \varphi_n \rangle &= \sum_{l\in\Gamma^+} u_l \int_0^H \phi_l(L,y) \varphi_n(y) dy \\
  &= \sum_{l\in\Gamma^+} Q_{nl} u_l, \label{eq:11b}
\end{align}
where
\begin{equation}
  \label{eq:13}
  Q_{nl} = \int_0^H \phi_l(L,y) \varphi_n(y) dy.
\end{equation}
From \eqref{eq:7}, \eqref{eq:2}, and \eqref{eq:11b}, we have
\begin{equation}
  \label{eq:14}
  a_j = \sum_{n=1}^N \varphi_{jn} \sum_{l\in\Gamma^+}P_{nl}u_l,
\end{equation}
where $\varphi_{jn} = \varphi_n(y_j)$ and $P_{nl} = I \beta_n Q_{nl}$. Replacing \eqref{eq:14} into \eqref{eq:10b},
\begin{equation}
  \label{eq:15}
  \int_{\Sigma_L}\Lambda_k^+(u) v dy = \sum_{i\in\Gamma^+}\sum_{j\in\Gamma^+} v_i C_{ij}^+ \sum_{n=1}^N \varphi_{jn} \sum_{l\in\Gamma^+}P_{nl}u_l
\end{equation}
The right-hand side of the previous equation can be written as a product of matrices, as desired, if the indices $i, j$, and $l$ are mapped  by an invertible mapping $r$ from $\Gamma^+$ to the set of consecutive indices $\{1,2,...,M\}$ with $M$ the cardinal of $\Gamma^+$. Let $m=r(i)$, $p=r(j)$, $q=r(l)$, $\ol{v}_m = v_i$, $\ol{C}_{mp}^+ = C_{ij}^+$, $\ol{\varphi}_{np} = \varphi_{nj}$, $\ol{P}_{nq} = P_{nl}$, and $\ol{u}_q = u_l$. In the new indices
\begin{align}
  \int_{\Sigma_L}\Lambda_k^+(u) v dy &= \sum_{m=1}^M\sum_{p=1}^M \ol{v}_m \ol{C}_{mp}^+ \sum_{n=1}^N \ol{\varphi}_{pn} \sum_{q=1}^M\ol{P}_{nq}\ol{u}_q \\
                                     &= {\bf v} C^+ \Phi P {\bf u} \\
  & = {\bf v}^T B {\bf u},
\end{align}
where ${\bf v} = \{\ol{v}_m\}$, $C^+ = \{\ol{C}_{mp}\}$, $\Phi=\{\ol{\varphi}_{pn}\}$, $P = \{\ol{P}_{nq}\}$, ${\bf u} = \{\ol{u}_q\}$, and
\begin{equation}
  \label{eq:26}
  B = C^+ \Phi P.
\end{equation}

% We would like to express the right-hand side of the Eq. \eqref{eq:15} as the product of a matrix by the vector ${\bf u}$ whose elementes are the $u$-values in the complete set of nodes, ${\bf u} = (u_1, u_2, \ldots, u_M)^T$. For that purpose, we define the following matrices: $\wt{C}^+$ is the $M \times M$ matrix with elements
% \begin{equation}
%   \label{eq:16}
%   \wt{C}^+_{ij} =
%   \begin{cases}
%     C_{ij}^+,~~i \in \Gamma^+ ~\mathrm{and}~ j \in \Gamma^+ \\
%     0,~~\mathrm{otherwise}
%   \end{cases}
% \end{equation}
% $\wt{\Phi}$ is the $M \times N$ matrix with elements
% \begin{equation}
%   \label{eq:17}
%   \wt{\Phi}_{ij} =
%   \begin{cases}
%     \varphi_{jn},~~j \in \Gamma^+ \\
%     0,~~\mathrm{otherwise}
%   \end{cases}
% \end{equation}
% and $\wt{P}_k$ is the $N \times M$ matrix with elements
% \begin{equation}
%   \label{eq:18}
%   \wt{P}_{k, nl} =
%   \begin{cases}
%     P_{nl},~~l \in \Gamma^+ \\
%     0,~~\mathrm{otherwise}
%   \end{cases}
% \end{equation}
% The subscript $k$ emphasizes the dependence of the matrix on the value of $k$ through the $\beta_n$, defined in Eq. \eqref{eq:3}.
% In terms of these matrices, Eq. \eqref{eq:10} becomes
% \begin{equation}
%   \label{eq:19}
%    \int_{\Sigma_L}\Lambda_k^+(u) v dy = {\bf v}^T \wt{C}^+ \wt{\Phi} \wt{P}_k {\bf u}.
% \end{equation}

To calculate the matrix elements $C_{ij}^+$ it is convenient to view the line segments between two adjacent points on $\Sigma_L$  as one-dimensional elements. Let $(L,y_e)$ and $(L,y_{e'})$, $y_e < y_{e'}$, be two adjacent points on $\Sigma_L$. We define the $e$-th one-dimensional element $\Omega_e$ as $\Omega_e = [y_e, y_{e'}]$. Therefore
\begin{align}
  \label{eq:20}
  C_{ij} &= \sum_{e \in \Gamma_e^+} \int_{\Omega_e} \phi_i(y)\phi_j(y) dy \\
         &= \sum_{e \in \Gamma_e^+} C_{ij}^{(e)},
\end{align}
where the dependence of the basis functions on $x$ has been removed for simplicity and $\Gamma_e^+$ is the set of indices of all the elements in $\Sigma_L$. The quantitiy $C_{ij}^{(e)}$ can be computed using the local coordinate $\xi$,
\begin{equation}
  \label{eq:yxi}
  y(\xi) = a_e + b_e\xi,
\end{equation}
where
\begin{equation}
  \label{eq:24}
  a_e =  \frac{1}{2}(y_{e'} + y_e)
\end{equation}
is the element midpoint and
\begin{equation}
  \label{eq:25}
  b_e = \frac{1}{2}(y_{e'} - y_e)
\end{equation}
is half the element length. By the change of variable \eqref{eq:yxi}, the basis functions $\phi_e(y)$ and $\phi_{e'}(y)$ become the local basis functions
\begin{align}
  \wt{\phi}_1(\xi) &= \frac{1}{2}(1-\xi) \\
  \wt{\phi}_2(\xi) &= \frac{1}{2}(1+\xi).
\end{align}
To the non-zero entries of the matrix $\left\{C_{ij}^{(e)}\right\}$ correspond a local matrix $\left\{\wt{C}_{rs}^{(e)}\right\}$, $r,s=1,2$, with entries $\wt{C}_{rs}^{(e)} = C_{ij}^{(e)}$, $i=q(e,r)$, $j=q(e,r)$, were $q$ is the mapping from local to global coordinates. In the local coordinate, the local-matrix entries can be computed as
\begin{align}
  \wt{C}_{rs}^{(e)} = b_e\int_{-1}^1 \wt{\phi}_r(\xi)\wt{\phi}_s(\xi)\,d\xi,~~r,s=1,2
\end{align}
yielding
\begin{equation}
  \label{eq:21}
  \left\{\wt{C}_{rs}^{(e)}\right\} = b_e
  \begin{bmatrix}
    \frac{2}{3} & \frac{1}{3} \\[1mm]
    \frac{1}{3} & \frac{2}{3}
  \end{bmatrix}.
\end{equation}

Similarly, the matrix entries $Q_{nl}$ defined in \eqref{eq:13}, can be obtained by a summation over the one-dimensional elements,
\begin{align}
  \label{eq:22}
  Q_{nl} &= \sum_{e \in \Gamma_e^+} \int_{\Omega_e} \phi_l(y) \sin(\lambda_n y)\, dy \\
         &= \sum_{e \in \Gamma_e^+} Q_{nl}^{(e)}.
\end{align}
The local counterpart of the matrix $\left\{Q_{nl}^{(e)}\right\}$ is the matrix $\left\{\wt{Q}_{nr}^{(e)}\right\}$, $r=1,2$, with entries
\begin{equation}
  \label{eq:23}
  \wt{Q}_{nr}^{(e)} = b_e \int_{-1}^1 \wt{\phi}_r(\xi)\sin[\lambda_n(a_e + b_e\xi)]\,d\xi,~~r = 1,2.
\end{equation}
The integrals above have analytical expression, leading to
\begin{align}
  Q_{n1}^{(e)} &= \frac{2 \, b_{e} \lambda_{n} \cos\left({\left(a_{e} - b_{e}\right)} \lambda_{n}\right) - \sin\left({\left(a_{e} + b_{e}\right)} \lambda_{n}\right) + \sin\left({\left(a_{e} - b_{e}\right)} \lambda_{n}\right)}{2 \, b_{e} \lambda_{n}^{2}} \\
  Q_{n2}^{(e)} &= -\frac{2 \, b_{e} \lambda_{n} \cos\left({\left(a_{e} + b_{e}\right)} \lambda_{n}\right) - \sin\left({\left(a_{e} + b_{e}\right)} \lambda_{n}\right) + \sin\left({\left(a_{e} - b_{e}\right)} \lambda_{n}\right)}{2 \, b_{e} \lambda_{n}^{2}}.
\end{align}




\clearpage
\bibliographystyle{unsrt}
\bibliography{references}

 \end{document}

  