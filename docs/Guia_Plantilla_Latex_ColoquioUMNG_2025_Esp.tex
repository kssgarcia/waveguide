\documentclass[11pt]{article}
\usepackage[utf8]{inputenc}
\usepackage[T1]{fontenc}
\usepackage[spanish]{babel}
\usepackage{hyperref}
\usepackage{graphicx}
\usepackage{setspace}
\usepackage{enumitem}
\usepackage{lipsum} % para texto de ejemplo
\usepackage{amsmath,amssymb}
\usepackage{ulem} % para \uline
\usepackage{fancyhdr}
\usepackage{xcolor}
\usepackage{hyperref}
\usepackage{makeidx}  % Necesario para generar el índice
\makeindex  % Habilita la creación del índice
\usepackage{eso-pic}
% Márgenes y paginación
\usepackage[a4paper, left=2.5cm, right=2.5cm, top=2.5cm, bottom=2.5cm]{geometry} % 2024
% Encabezados y pies
\pagestyle{fancy}
\fancyhf{}
\fancyhead[LE,RO]{\thepage}
\fancyhead[RE]{\leftmark}
\fancyhead[LO]{\rightmark}
% Comandos para ponente subrayado en autores
\newcommand{\ponente}[1]{\underline{#1}}
\let\cleardoublepage\clearpage
\begin{document}
\cleardoublepage
\begin{center}
	\textbf{\LARGE{XI COLOQUIO DE MATEM\'ATICAS APLICADAS }}\\
    {\Large Universidad Militar Nueva Granada, Colombia - 2025 \par}
	\vspace{0.4cm}
\end{center}
\section*{Título de la ponencia o del poster}
\addcontentsline{toc}{section}{Modelado Estadístico de Incertidumbre en Sistemas Dinámicos}
%
\thispagestyle{plain}
\vspace*{1cm}
\begin{center}
	{\large Primer Autor\textsuperscript{1}, \ponente{Autor Ponente}\textsuperscript{2}, Tercer Autor \textsuperscript{3}}\\[1em]
\end{center}

\footnotetext[1]{Universidad de la República, Uruguay. \texttt{cduarte@udelar.edu.uy}}
\footnotetext[2]{Universidad de los Andes, Colombia. \texttt{erivas@uniandes.edu.co}}
\footnotetext[3]{Universidade de São Paulo, Brasil. \texttt{jteixeira@usp.br}}

\noindent\textbf{Resumen:}\\
Por favor escriba aquí el resumen o abstract de su charla o poster. \textbf{Por favor NO cambie el formato a begin abstract.} El contenido de todo el documento debe ser mínimo de 2 páginas y máximo 3. 

\medskip
\noindent\textbf{Palabras Claves}: Máximo 6 palabras claves.

\vspace{1em}
\noindent\textbf{Introducción}\\
Desde aquí puede escribir una breve descripción del contenido de su charla.

La Figura~\ref{tab:comparacion_bayes} es un ejemplo para insertar figuras en el documento. \textbf{Por favor al final de la descripción de la gráfica, escirba:} Fuente: Figura realizada por el autor.



\begin{figure}[h!]
	\centering
	% \includegraphics[width=\linewidth]{LogoColoquio2025.eps}
	\caption{Al final de la descripción de su gráfica escriba: Fuente: Figura realizada por el autor.}
	\label{fig:bayes_poblacion}
\end{figure}





\vspace{1em}
\noindent\textbf{Ejemplo con tablas}\\
La Tabla~\ref{tab:comparacion_bayes} es un ejemplo para describir las tablas en su documento.


\begin{table}[h!]
	\centering
	\begin{tabular}{|p{4.2cm}|c|c|c|}
		\hline
		\textbf{Método} & \textbf{Precisión} & \textbf{Costo computacional} & \textbf{Incertidumbre} \\
		\hline
		Ajuste clásico (mínimos cuadrados) & Media & Bajo & Bajo \\
		MCMC bayesiano & Alta & Alto & Excelente \\
		Filtrado de partículas & Alta & Medio & Bueno \\
		\hline
	\end{tabular}
	\caption{Comparación de métodos para inferencia de parámetros en sistemas dinámicos.}
	\label{tab:comparacion_bayes}
\end{table}

\vspace{1em}
\noindent\textbf{Conclusión}\\
\textbf{Por favor agregar  Conclusiones} Recuerde que el documento debe tener un mínimo de 2 páginas y máximo 3. \textbf{Por favor no cambie el formato de la bibliografía, use el que se ha indicado en este archivo .TEX}
 
\vspace{1em}
\noindent\textbf{Bibliografía}
\begin{itemize}[leftmargin=1.5em]
	\item C. Andrieu, A. Doucet, R. Holenstein, \textit{Particle Markov chain Monte Carlo methods}, Journal of the Royal Statistical Society: Series B, \textbf{72}(3), 269--342, 2010.
	\item M. Girolami, B. Calderhead, \textit{Riemann manifold Langevin and Hamiltonian Monte Carlo methods}, Journal of the Royal Statistical Society: Series B, \textbf{73}(2), 123--214, 2011.
	\item H. Rue, S. Martino, N. Chopin, \textit{Approximate Bayesian inference for latent Gaussian models by using integrated nested Laplace approximations}, Journal of the Royal Statistical Society: Series B, \textbf{71}(2), 319--392, 2009.
\end{itemize}	
\end{document}
