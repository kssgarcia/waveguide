\documentclass[11pt]{article}
% \setlength{\voffset}{-2cm}
% \setlength{\hoffset}{-2cm}
% \setlength{\textwidth}{16cm}
% \setlength{\textheight}{25cm}
% \setlength{\marginparwidth}{1cm}
\usepackage[english,spanish,activeacute]{babel}
\usepackage{graphicx,array}
\usepackage{amsmath}
%\usepackage{pstricks}
\usepackage{multicol}
\usepackage{color}
\usepackage{listings}
\usepackage[inline]{enumitem}
\usepackage[colorlinks]{hyperref}
% \usepackage[hypertexnames=false]{hyperref}
% \pagestyle{empty}
\decimalpoint

\definecolor{darkgray}{rgb}{0.66, 0.66, 0.66}
\definecolor{darkorange}{rgb}{1.0, 0.55, 0.0}
\definecolor{gray}{rgb}{0.97,0.97,0.99}
\definecolor{teal}{rgb}{0.0, 0.5, 0.5}
\definecolor{comment}{rgb}{0.6, 0, 0.9}
\lstdefinestyle{mystyle}{
	language = Python,
	backgroundcolor=\color{gray},
	commentstyle=\color{comment},
	keywordstyle=\bfseries\color{darkorange},
	numberstyle=\scriptsize\color{darkgray},
	stringstyle=\color{teal},
	basicstyle=\scriptsize\ttfamily,%\linespread{1}
	breakatwhitespace=false,
	breaklines=true,
	captionpos=t,
	keepspaces=false,
	numbers=left,
	numbersep=3pt,
	showspaces=false,
	showstringspaces=false,
	showtabs=false,
	tabsize=4
}
\lstset{style=mystyle}


\begin{document}
\begin{center}
{\Large Soluci\'on de la ecuaci\'on de onda unidimensional}
\end{center}

El estudiante debe estar en capacidad de resolver estos problemas, entender
las diferencias en cada caso y explicar las implicaciones f\'isicas que las condiciones
de frontera impone.

\section*{Problema 1}
\noindent Resolver la ecuaci\'on de onda unidimensional:
\begin{equation}
  \label{eq:onda}
  u_{tt} - c^2 u_{xx} = 0,~~0 \le x \le 1,~~t>0
\end{equation}
sujeta a las condiciones de frontera (Dirichlet):
\begin{equation}
  \label{eq:dirich}
  u(0,t) = u(1,t) = 0,
\end{equation}
y a las condiciones iniciales
\begin{equation}
  \label{eq:6}
  u(x,0) = u_0(x),~~u_t(x,0)=u_1(x).
\end{equation}

{\bf Soluci\'on anal\'itica:} Se usar\'a el m\'etodo de separaci\'on de variables, en el cual se supone que la soluci\'on tiene la forma
\begin{equation}
  \label{eq:wv}
  u(x,t) = w(t)v(x).
\end{equation}
Remplazando \eqref{eq:wv} en \eqref{eq:onda} y pasando el segundo t\'ermino a la derecha, tenemos
\begin{equation}
  w''(t)v(x) = c^2w(t)v''(x). \\
\end{equation}
Dividiendo entre $w(t)v(x)$,
\begin{equation}
  \label{eq:lambda}
    \frac{w''(t)}{w(t)} = c^2\frac{v''(x)}{v(x)} = \lambda,
\end{equation}
donde se est\'a  indicando que para que una funci\'on de $t$ sea igual a una funci\'on de $x$, ambas funciones deben ser iguales a una constante, llamada $\lambda$. De la ecuaci\'on \eqref{eq:lambda} se deducen las ecuaciones diferenciales que deben obedecer $w(t)$ y $v(x)$,
\begin{eqnarray}
  \label{eq:4}
  w'' - \lambda w  &= 0 \\
  v'' - \frac{\lambda}{c^2}v &= 0.
\end{eqnarray}
Las soluciones $w(t)$ y $v(t)$ dependen del signo de $\lambda$ como se indica en el Cuadro \ref{tab:sol}


\begin{table}[h]
  \centering
  \begin{tabular}{|c|c|c|c} \hline
    $\lambda$ &  $w(t)$ & $v(x)$  \\ \hline
    $\lambda = -\omega^2 < 0$ & $\cos(\omega t), \sin(\omega t)$ & $\cos\left(\frac{\omega x}{c}\right), \sin\left(\frac{\omega x}{c}\right)$ \\ \hline
    $\lambda = 0$ & $1, t$ & $1, x$ \\ \hline
    $\lambda = \omega^2 < 0$ & $e^{-\omega t}, e^{\omega t}$ & $e^{-\omega x/c}, e^{\omega x/c}$ \\ \hline
  \end{tabular}
  \caption{Soluciones $w(t)$ y $v(x)$.}
  \label{tab:sol}
\end{table}
Las condiciones de frontera $u(0,t) = w(t)v(0) = 0$ y $u(1,t) = w(t)v(1)=0$, implican que $v(0)=v(1)=0$. Del Cuadro \ref{tab:sol}, esta condici\'on solo la satisfacen las funciones
\begin{equation}
  \label{eq:2}
v_n(x) = \sin(\omega_n x/c),~~\omega_n = n\pi c,~~n=1,2,3,...
\end{equation}
A estas funciones corresponden soluciones de \eqref{eq:onda} de la forma
\begin{equation}
  \label{eq:3}
  u_n(x,t) = \cos(n\pi c t)\sin(n\pi x),~~ \tilde{u}_n(x,t) = \sin(n\pi c t)\sin(n\pi x),~~n=1,2,3,...
\end{equation}
La soluci\'on general de \eqref{eq:onda} es una superposici\'on de las soluciones $u_n$ y $\tilde{u}_n$, conocida como una serie de Fourier
\begin{equation}
  \label{eq:5}
  u(x,t) = \sum_{n=1}^\infty \left[b_n\cos(n\pi c t) + d_n\sin(n\pi c t)\right] \sin(n\pi x).
\end{equation}
Los coeficientes $b_n$ y $d_n$ se pueden calcular a partir de las condiciones iniciales,
\begin{equation}
  \label{eq:7}
  u(x,0) = u_0(x) = \sum_{n=1}^\infty b_n\sin(n\pi x).
\end{equation}

\begin{equation}
  \label{eq:7}
  u_t(x,0) = u_1(x) = \sum_{n=1}^\infty n\pi c d_n \sin(n\pi x).
\end{equation}
Usando la ortogonalidad de las funciones $\sin(n\pi x)$ se tiene que
\begin{eqnarray}
  \label{eq:8}
  b_n &= 2 \displaystyle\int_0^1 u_0(x)\sin(n\pi x) \\
  d_n &= \displaystyle\frac{2}{n\pi c}\int_0^1 u_1(x)\sin(n\pi x).
\end{eqnarray}

{\bf Ejemplo:} A manera de ilustraci\'on, consideramos el caso en que $c=1$,
\begin{equation}
  \label{eq:9}
  u_0(x) =
  \begin{cases}
    x,~~ 0 \le x \le \frac{1}{2}, \\
    1-x,~~ \frac{1}{2} \le x \le 1
  \end{cases}
\end{equation}
y $u_1(x) = 0$.
Al evaluar las integrales se obtiene
\begin{equation}
  \label{eq:10}
  b_n =
  \begin{cases}
    \frac{4(-1)^k}{(2k+1)^2\pi^2},~~n=2k+1 \\
    0,~~n=2k,
  \end{cases}
\end{equation}
mientras que $d_n=0$.
Por lo tanto la soluci\'on de \eqref{eq:onda} es
\begin{equation}
  \label{eq:tienda}
  u(x,t) = \frac{4}{\pi^2} \sum_{k=0}^\infty \frac{(-1)^k}{(2k+1)^2} \cos[(2k+1)\pi t] \sin[(2k+1)\pi x].
\end{equation}
Esta soluci\'on, con la serie truncada a 100 t\'erminos, se presenta en la Fig. \ref{fig:tienda} para varios tiempos $t$ (la soluci\'on fue calculada con el Listing \ref{code:fourier_dir}).
\begin{figure}[h]
  \centering
  \includegraphics[width=6cm]{t_0.0.pdf} 
  \includegraphics[width=6cm]{t_0.25.pdf} \\
  \includegraphics[width=6cm]{t_0.5.pdf} 
  \includegraphics[width=6cm]{t_0.75.pdf} \\
  \includegraphics[width=6cm]{t_1.0.pdf}
  \includegraphics[width=6cm]{t_1.25.pdf}
  \caption{Soluci\'on $u(x,t)$ presentada en la Ec. \eqref{eq:tienda}, con la serie truncada a 100 t\'erminos, para varios tiempos.}
  \label{fig:tienda}
\end{figure}

\lstinputlisting[language=Python, caption=Soluci\'on anal\'itica de la ecuaci\'on de onda, label=code:fourier_dir]{fourier.py}


{\bf Soluci\'on num\'erica:} La soluci\'on num\'erica se calcular\'a mediante diferencias finitas, usando la relaci\'on de recurrencia
\begin{equation}
  \label{eq:1}
  {\bf u}^{(j+1)} = B{\bf u}^{(j)} - u^{(j-1)} + {\bf b}^{(j)},
\end{equation}
donde
\begin{equation}
  \label{eq:11}
  B =
  \begin{bmatrix}
    2(1-\sigma^2) & \sigma^2      &                      \\
    \sigma^2      & 2(1-\sigma^2) & \sigma^2             \\
                  & \sigma^2      & \ddots     & \ddots  \\
                  &               & \ddots     & \sigma^2 \\
                  &               & \sigma^2   & 2(1-\sigma^2)
  \end{bmatrix},
  {\bf u}^{(j)} =
  \begin{bmatrix}
    u_{j,1} \\
    u_{j,2} \\
    \vdots \\
    u_{j,n-2} \\
    u_{j,n-1}
  \end{bmatrix},
  {\bf b}^{(j)} =
  \begin{bmatrix}
    \sigma^2\alpha_j \\
    0 \\
    \vdots \\
    0 \\
    \sigma^2\beta_j
  \end{bmatrix},
\end{equation}
con
\begin{equation}
  \label{eq:12}
  \sigma = c\Delta t/\Delta x,
\end{equation}
y $u_{j,m} \approx u(m\Delta x, j\Delta t)$. Para iniciar la iteraci\'on, ${\bf u}^{(0)}$ y ${\bf u}^{(1)}$ se asignan de la siguiente forma. $u_{0,m} = u_0(m\Delta x)$ y 
\begin{equation}
  \label{eq:13}
  {\bf u}^{(1)} = \frac{1}{2}B{\bf u}^{(0)} + {\bf g}\Delta t + \frac{1}{2}{\bf b}^{(0)},
\end{equation}
donde ${\bf g}$ tiene elementos $g_m = u_1(m\Delta x)$. La Fig. \ref{fig:num} compara la soluci\'on anal\'itica con la soluci\'on  num\'erica calculada con $\Delta x=\Delta t=0.01$ usando el Listing \ref{code:onda_dirich}.
\begin{figure}[h]
  \centering
    \includegraphics[width=6cm]{num_t_0.0.pdf} 
  \includegraphics[width=6cm]{num_t_0.25.pdf} \\
  \includegraphics[width=6cm]{num_t_0.5.pdf} 
  \includegraphics[width=6cm]{num_t_0.75.pdf} \\
  \includegraphics[width=6cm]{num_t_1.0.pdf}
  \includegraphics[width=6cm]{num_t_1.25.pdf}
  \caption{Comparaci\'on de la soluci\'on num\'erica con la soluci\'on anal\'itica de \eqref{eq:onda}.}
  \label{fig:num}
\end{figure}

\lstinputlisting[language=Python, caption=Soluci\'on num\'erica de la ecuaci\'on de onda, label=code:onda_dirich]{numerical.py}

\clearpage
\section*{Problema 2}
Resolver la ecuaci\'on de onda unidimensional:
\begin{equation}
  \label{eq:onda_nh}
  u_{tt} - c^2 u_{xx} = f(x,t),~~0 \le x \le 1,~~t>0
\end{equation}
sujeta a las condiciones de frontera (Dirichlet):
\begin{equation}
  \label{eq:dirich}
  u(0,t) = u(1,t) = 0,
\end{equation}
y a las condiciones iniciales
\begin{equation}
  \label{eq:6}
  u(x,0) = 0,~~u_t(x,0)=0.
\end{equation}

{\bf Soluci\'on anal\'itica:} Este problema puede resolverse por el m\'etodo de expansi\'on en eigenfunciones. La soluci\'on $u(x,t)$ y la funci\'on de forzamiento $f(x,t)$ pueden expresarse en t\'erminos de las eigenfuciones $v_n(x)$ as\'i
\begin{align}
  \label{eq:uxt}
u(x,t) &= \displaystyle\sum_{n=1}^\infty w_n(t)v_n(x) \\
       &= \displaystyle\sum_{n=1}^\infty w_n(t)\sin(n\pi x). 
\end{align}
y
\begin{equation}
  \label{eq:fxt}
  f(x,t) = \sum_{n=1}^\infty F_n(t)\sin(n\pi x),
\end{equation}
donde
\begin{equation}
  \label{eq:16}
  F_n(t) = 2 \int_0^1 f(x,t)\sin(n\pi x) dx,~~n=1,2,3,...
\end{equation}

Al remplazar \eqref{eq:uxt} y \eqref{eq:fxt} en \eqref{eq:onda_nh} tenemos
\begin{equation}
  \label{eq:14}
  \sum_{n=1}^\infty \left[w''_n(t) + n^2\pi^2c^2 w_n(t)\right]\sin(n\pi x) = \sum_{n=1}^\infty F_n(t)\sin(n\pi x).
\end{equation}
Usando la ortogonalidad de las funciones $\sin(n\pi x)$ se llega a
\begin{equation}
  \label{eq:15}
  w''_n + (n\pi c)^2 w_n = F_n(t),~~n=1,2,3,...
\end{equation}
La ecuaciones diferenciales ordinarias anteriores tienen soluciones 
\begin{equation}
  \label{eq:17}
  w_n(t) = A_n \cos(n\pi ct) + B_n \sin(n\pi ct) + \frac{1}{n\pi c}\int_0^t \sin[n\pi c(t-s)]F_n(s)ds.
\end{equation}
En $t=0$,
\begin{equation}
  \label{eq:18}
  u(x,0) = \sum_{n=1}^\infty w_n(0)\sin(n\pi x) = 0,
\end{equation}
y
\begin{equation}
  \label{eq:18}
  u_t(x,0) = \sum_{n=1}^\infty w_n'(0)\sin(n\pi x) = 0,
\end{equation}
por lo cual $w_n(0) = w'_n(0)=0$. Las condiciones anteriores implican $A_n = B_n = 0$. Como resultado, la soluci\'on general se reduce a
\begin{equation}
  \label{eq:20}
  u(x,t) = \sum_{n=1}^\infty \frac{1}{n\pi c} \sin(n\pi x) \int_0^t \sin[n\pi c(t-s)]F_n(s)ds.
\end{equation}

{\bf Ejemplo:} Supongamos que $f(x,t) = \sin(t)\sin(\pi x)$. Es decir $F_1(t) = \sin(t)$, $F_n(t) = 0$, $n=2,3,...$. En este caso la soluci\'on \eqref{eq:20} se reduce a 
\begin{equation}
  \label{eq:21}
  u(x,t) =  \left[-\frac{1}{(\pi c)^3 - \pi c}\sin(\pi ct) +
    \frac{1}{(\pi c)^2 - 1}\sin t \right] \sin(\pi x).
\end{equation}

{\bf Soluci\'on num\'erica:} La soluci\'on num\'erica se puede calcular mediante la siguiente variaci\'on de la relaci\'on de recurrencia expresada en la Eq. \eqref{eq:1},
\begin{equation}
  \label{eq:22}
    {\bf u}^{(j+1)} = B{\bf u}^{(j)} + {\bf f}^{(j)}- u^{(j-1)} + {\bf b}^{(j)},
\end{equation}
donde ${\bf f}^{(j)}$ tiene elementos $f_{j,m} = f(m\Delta x, j\Delta t)$. La Fig. \ref{fig:num_nh_dir}, generada con el Listing \eqref{code:onda_nh_dir}, compara la soluci\'on anal\'itica \eqref{eq:21} con la soluci\'on num\'erica. 

\begin{figure}[h]
  \centering
    \includegraphics[width=6cm]{num_t_nh_0.0.pdf} 
  \includegraphics[width=6cm]{num_t_nh_1.2.pdf} \\
  \includegraphics[width=6cm]{num_t_nh_2.4.pdf} 
  \includegraphics[width=6cm]{num_t_nh_3.6.pdf} \\
  \includegraphics[width=6cm]{num_t_nh_4.8.pdf}
  \includegraphics[width=6cm]{num_t_nh_6.0.pdf}
  \caption{Comparaci\'on de la soluci\'on num\'erica con la soluci\'on anal\'itica de \eqref{eq:onda_nh}.}
  \label{fig:num_nh_dir}
\end{figure}

\lstinputlisting[language=Python, caption=Comparaci\'on de la soluci\'on anal\'itica y la soluci\'on num\'erica de \eqref{eq:onda_nh}, label=code:onda_nh_dir]{numerical_nh_dir.py}


\clearpage
\section*{Problema 3}
Resolver la ecuaci\'on de onda unidimensional:
\begin{equation}
  \label{eq:onda_nh_bc}
  u_{tt} - c^2 u_{xx} = 0,~~0 \le x \le 1,~~t>0
\end{equation}
sujeta a las condiciones de frontera (Dirichlet):
\begin{equation}
  \label{eq:nh_dirich}
  u(0,t) = g(t),~~u(1,t) = 0,
\end{equation}
y a las condiciones iniciales
\begin{equation}
  \label{eq:6}
  u(x,0) = 0,~~u_t(x,0)=0.
\end{equation}
{\bf Soluci\'on anal\'itica:} La soluci\'on $u(x,t)$ puede escribirse como la suma de dos funciones
\begin{equation}
  \label{eq:23}
  u(x,t) = \phi(x,t) + \psi(x,t),
\end{equation}
donde $\psi(x,t)$ satisface las condiciones de frontera no homog\'eneas,
\begin{equation}
  \label{eq:28}
  \psi(x,0) = g(t),~~\psi(x,1) = 0,
\end{equation}
y, como veremos en seguida, $\phi$ es la soluci\'on de un problema con condiciones de frontera homog\'eneas. Una elecci\'on v\'alida para $\psi$ es
\begin{equation}
  \label{eq:31}
  \psi(x,t) = g(t)(1-x).
\end{equation}

Remplazando \eqref{eq:23} en \eqref{eq:onda_nh_bc},
\begin{equation}
  \label{eq:24}
  \phi_{tt} + \psi_{tt} - c^2(\phi_{xx} + \psi_{xx}) = 0.
\end{equation}
y reordenando t\'erminos,
\begin{equation}
  \label{eq:25}
  \phi_{tt} - c^2\phi_{xx} = f(x,t),
\end{equation}
donde
\begin{align}
  \label{eq:26}
  f(x,t) &= -\psi_{tt} + c^2\psi_{xx} \\
         &= g''(t)(x-1).
\end{align}

Las condiciones de frontera de $\phi$ est\'an determinadas por las condiciones de frontera de $u(x,t)$,
\begin{align}
  \label{eq:27}
  u(0,t) &= \phi(0,t) + \psi(0,t) = g(t) \\
  u(1,t) &= \phi(1,t) + \psi(1,t) = 0.
\end{align}
Es decir, $\phi$ debe cumplir condiciones de frontera homog\'eneas,
\begin{equation}
  \label{eq:29}
  \phi(0,t) = \phi(1,t) = 0.
\end{equation}
An\'alogamente, las condiciones iniciales para $\phi$ se pueden determinar con base en las condiciones iniciales de $u(x,t)$,
\begin{align}
  \label{eq:30}
  u(x,0) &= \phi(x,0) + \psi(x,0) = 0 \\
  u_t(x,0) &= \phi_t(x,0) + \psi_t(x,0) = 0.
\end{align}
Es decir,
\begin{subequations}
\label{eq:phi_ic}
\begin{align}
  \phi(x,0) &= -\psi(x,0) = g(0)(x-1) \\
  \phi_t(x,0) &= -\psi_t(x,0) = g'(0)(x-1).
\end{align}  
\end{subequations}

En resumen, $\phi$ debe cumplir el problema determinado por las ecuaciones \eqref{eq:25}, \eqref{eq:29} y \eqref{eq:phi_ic}, el cual es del mismo tipo que el Problema 2 (secci\'on anterior) y puede resolverse mediante el procedimiento presentado en esa secci\'on.

{\bf Ejemplo:} Sea $g(t) = \sin(t)$. Entonces,

\begin{align}
  \label{eq:32}
  f(x,t) &= \sin(t)(1-x) \\
         &= \sin(t) \sum_{n=1}^\infty \frac{2}{n\pi}\sin(n\pi x).
\end{align}
De acuerdo con la soluci\'on del Problema 2, la funci\'on $\phi(x,t)$ tiene expansi\'on en serie de Fourier
\begin{equation}
  \label{eq:34}
  \phi(x, t) = \sum_{n=1}^\infty w_n(t)\sin(n\pi x),
\end{equation}
donde los coeficientes $w_n(t)$ son soluciones de \eqref{eq:15} con las funciones de forzamiento $F_n(t)$ determinadas por la expansi\'on en serie de Fourier de $f(x,t)$,
\begin{equation}
f(x,t) = \sum_{n=1} F_n(t) \sin(n\pi x).
\end{equation}
Es decir
\begin{equation}
  \label{eq:35}
  F_n(t) = \frac{2}{n\pi} \sin(t).
\end{equation}
La soluciones $w_n(t)$ pueden calcularse mediante la Ec. \eqref{eq:17}. Para hallar los coeficientes $A_n$ y $B_n$ usamos las condiciones iniciales \eqref{eq:phi_ic}, que se reducen a
\begin{subequations}
  \begin{align}
    \phi(x,0) &= 0\\
    \phi_t(x,0) &= x-1.
  \end{align}
\end{subequations}
Por lo tanto $A_n = 0$ y $B_n$ se determina mediante,
\begin{equation}
  \label{eq:40}
  \phi_t(x,0) = \sum_{n=1}^\infty n\pi c B_n \sin(n\pi x) = -\sum_{n=1}^\infty \frac{2}{n\pi} \sin(n\pi x),
\end{equation}
dando como resultado
\begin{equation}
  \label{eq:41}
  B_n = -\frac{2}{(n\pi)^2 c}.
\end{equation}

En consecuencia, los coeficientes $w_n(t)$ en \eqref{eq:34} toman la forma,
\begin{equation}
  w_n(t) = \frac{2}{n\pi}\left[\left( -\frac{1}{n\pi c} -\frac{1}{(n\pi c)^3 - (n\pi c)} \right)\sin(n\pi ct)
    + \frac{1}{(n\pi c)^2-1}\sin(t) \right]
\end{equation}
Finalmente, la soluci\'on del problema original $u(x,t)$ se obtiene remplazando \eqref{eq:34} y \eqref{eq:31} en \eqref{eq:23}
\begin{equation}
  \label{eq:39}
    u(x,t) = \sin(t)(1-x) + \sum_{n=1}^\infty w_n(t) \sin(n\pi x).
\end{equation}

{\bf Soluci\'on num\'erica:} La soluci\'on num\'erica se puede calcular con la relaci\'on de recurrencia \eqref{eq:1}, con $\alpha_j = \sin(j\Delta t)$ en ${\bf b}^{(j)}$. La Fig. \ref{fig:num_nh_bc}, generada con el Listing \ref{code:num_nh_bc}, compara la soluci\'on anal\'itica con la soluci\'on num\'erica para diferentes tiempos.

\begin{figure}[h]
  \centering
    \includegraphics[width=6cm]{num_nh_bc_t_0.0.pdf} 
  \includegraphics[width=6cm]{num_nh_bc_t_1.2.pdf} \\
  \includegraphics[width=6cm]{num_nh_bc_t_2.4.pdf} 
  \includegraphics[width=6cm]{num_nh_bc_t_3.6.pdf} \\
  \includegraphics[width=6cm]{num_nh_bc_t_4.8.pdf}
  \includegraphics[width=6cm]{num_nh_bc_t_6.0.pdf}
  \caption{Comparaci\'on de la soluci\'on num\'erica con la soluci\'on anal\'itica del problema determinado por \eqref{eq:onda_nh_bc}, \eqref{eq:nh_dirich} y \eqref{eq:6}.}
  \label{fig:num_nh_bc}
\end{figure}

\lstinputlisting[language=Python, caption={Comparaci\'on de la soluci\'on anal\'itica y la soluci\'on num\'erica del problema determinado por \eqref{eq:onda_nh_bc}, \eqref{eq:nh_dirich} y \eqref{eq:6}}, label=code:num_nh_bc]{numerical_nh_bc.py}
\end{document}

